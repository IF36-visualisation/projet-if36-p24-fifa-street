% Options for packages loaded elsewhere
\PassOptionsToPackage{unicode}{hyperref}
\PassOptionsToPackage{hyphens}{url}
%
\documentclass[
]{article}
\usepackage{amsmath,amssymb}
\usepackage{iftex}
\ifPDFTeX
  \usepackage[T1]{fontenc}
  \usepackage[utf8]{inputenc}
  \usepackage{textcomp} % provide euro and other symbols
\else % if luatex or xetex
  \usepackage{unicode-math} % this also loads fontspec
  \defaultfontfeatures{Scale=MatchLowercase}
  \defaultfontfeatures[\rmfamily]{Ligatures=TeX,Scale=1}
\fi
\usepackage{lmodern}
\ifPDFTeX\else
  % xetex/luatex font selection
\fi
% Use upquote if available, for straight quotes in verbatim environments
\IfFileExists{upquote.sty}{\usepackage{upquote}}{}
\IfFileExists{microtype.sty}{% use microtype if available
  \usepackage[]{microtype}
  \UseMicrotypeSet[protrusion]{basicmath} % disable protrusion for tt fonts
}{}
\makeatletter
\@ifundefined{KOMAClassName}{% if non-KOMA class
  \IfFileExists{parskip.sty}{%
    \usepackage{parskip}
  }{% else
    \setlength{\parindent}{0pt}
    \setlength{\parskip}{6pt plus 2pt minus 1pt}}
}{% if KOMA class
  \KOMAoptions{parskip=half}}
\makeatother
\usepackage{xcolor}
\usepackage[margin=1in]{geometry}
\usepackage{color}
\usepackage{fancyvrb}
\newcommand{\VerbBar}{|}
\newcommand{\VERB}{\Verb[commandchars=\\\{\}]}
\DefineVerbatimEnvironment{Highlighting}{Verbatim}{commandchars=\\\{\}}
% Add ',fontsize=\small' for more characters per line
\usepackage{framed}
\definecolor{shadecolor}{RGB}{248,248,248}
\newenvironment{Shaded}{\begin{snugshade}}{\end{snugshade}}
\newcommand{\AlertTok}[1]{\textcolor[rgb]{0.94,0.16,0.16}{#1}}
\newcommand{\AnnotationTok}[1]{\textcolor[rgb]{0.56,0.35,0.01}{\textbf{\textit{#1}}}}
\newcommand{\AttributeTok}[1]{\textcolor[rgb]{0.13,0.29,0.53}{#1}}
\newcommand{\BaseNTok}[1]{\textcolor[rgb]{0.00,0.00,0.81}{#1}}
\newcommand{\BuiltInTok}[1]{#1}
\newcommand{\CharTok}[1]{\textcolor[rgb]{0.31,0.60,0.02}{#1}}
\newcommand{\CommentTok}[1]{\textcolor[rgb]{0.56,0.35,0.01}{\textit{#1}}}
\newcommand{\CommentVarTok}[1]{\textcolor[rgb]{0.56,0.35,0.01}{\textbf{\textit{#1}}}}
\newcommand{\ConstantTok}[1]{\textcolor[rgb]{0.56,0.35,0.01}{#1}}
\newcommand{\ControlFlowTok}[1]{\textcolor[rgb]{0.13,0.29,0.53}{\textbf{#1}}}
\newcommand{\DataTypeTok}[1]{\textcolor[rgb]{0.13,0.29,0.53}{#1}}
\newcommand{\DecValTok}[1]{\textcolor[rgb]{0.00,0.00,0.81}{#1}}
\newcommand{\DocumentationTok}[1]{\textcolor[rgb]{0.56,0.35,0.01}{\textbf{\textit{#1}}}}
\newcommand{\ErrorTok}[1]{\textcolor[rgb]{0.64,0.00,0.00}{\textbf{#1}}}
\newcommand{\ExtensionTok}[1]{#1}
\newcommand{\FloatTok}[1]{\textcolor[rgb]{0.00,0.00,0.81}{#1}}
\newcommand{\FunctionTok}[1]{\textcolor[rgb]{0.13,0.29,0.53}{\textbf{#1}}}
\newcommand{\ImportTok}[1]{#1}
\newcommand{\InformationTok}[1]{\textcolor[rgb]{0.56,0.35,0.01}{\textbf{\textit{#1}}}}
\newcommand{\KeywordTok}[1]{\textcolor[rgb]{0.13,0.29,0.53}{\textbf{#1}}}
\newcommand{\NormalTok}[1]{#1}
\newcommand{\OperatorTok}[1]{\textcolor[rgb]{0.81,0.36,0.00}{\textbf{#1}}}
\newcommand{\OtherTok}[1]{\textcolor[rgb]{0.56,0.35,0.01}{#1}}
\newcommand{\PreprocessorTok}[1]{\textcolor[rgb]{0.56,0.35,0.01}{\textit{#1}}}
\newcommand{\RegionMarkerTok}[1]{#1}
\newcommand{\SpecialCharTok}[1]{\textcolor[rgb]{0.81,0.36,0.00}{\textbf{#1}}}
\newcommand{\SpecialStringTok}[1]{\textcolor[rgb]{0.31,0.60,0.02}{#1}}
\newcommand{\StringTok}[1]{\textcolor[rgb]{0.31,0.60,0.02}{#1}}
\newcommand{\VariableTok}[1]{\textcolor[rgb]{0.00,0.00,0.00}{#1}}
\newcommand{\VerbatimStringTok}[1]{\textcolor[rgb]{0.31,0.60,0.02}{#1}}
\newcommand{\WarningTok}[1]{\textcolor[rgb]{0.56,0.35,0.01}{\textbf{\textit{#1}}}}
\usepackage{graphicx}
\makeatletter
\def\maxwidth{\ifdim\Gin@nat@width>\linewidth\linewidth\else\Gin@nat@width\fi}
\def\maxheight{\ifdim\Gin@nat@height>\textheight\textheight\else\Gin@nat@height\fi}
\makeatother
% Scale images if necessary, so that they will not overflow the page
% margins by default, and it is still possible to overwrite the defaults
% using explicit options in \includegraphics[width, height, ...]{}
\setkeys{Gin}{width=\maxwidth,height=\maxheight,keepaspectratio}
% Set default figure placement to htbp
\makeatletter
\def\fps@figure{htbp}
\makeatother
\setlength{\emergencystretch}{3em} % prevent overfull lines
\providecommand{\tightlist}{%
  \setlength{\itemsep}{0pt}\setlength{\parskip}{0pt}}
\setcounter{secnumdepth}{-\maxdimen} % remove section numbering
\ifLuaTeX
  \usepackage{selnolig}  % disable illegal ligatures
\fi
\IfFileExists{bookmark.sty}{\usepackage{bookmark}}{\usepackage{hyperref}}
\IfFileExists{xurl.sty}{\usepackage{xurl}}{} % add URL line breaks if available
\urlstyle{same}
\hypersetup{
  pdftitle={Analyse de la Premier League},
  pdfauthor={Groupe Fifa Street},
  hidelinks,
  pdfcreator={LaTeX via pandoc}}

\title{Analyse de la Premier League}
\author{Groupe Fifa Street}
\date{}

\begin{document}
\maketitle

\hypertarget{introduction}{%
\section{Introduction}\label{introduction}}

Nous avons décidé d'orienter notre projet sur la Premier League, qui est
la première division de football en Angleterre. La Premier League fait
partie des 5 grands championnats avec la Ligue 1 (France), la Liga
(Espagne), la Bundesliga (Allemagne) et la Serie A (Italie).

Nous allons utiliser une librairie qui contient une multitude de
fonctions qui requièrent, en fonction des requêtes, une des 3 API des
sites suivants : FBref, Transfermarkt et Understat. Le lien qui décrit
la librairie et toutes ses fonctionnalités est le suivant :
\href{https://jaseziv.github.io/worldfootballR/articles/extract-fbref-data.html}{worldfootballR}

Les sources sont tirées de FBref, Transfermarkt et Understat.

\textbf{FBref} est un site web qui fournit des statistiques détaillées
sur les joueurs et les équipes de football, y compris des données sur
les passes, les tirs et les actions défensives.

\textbf{Transfermarkt} est un site web qui se concentre sur les
transferts de joueurs de football, les évaluations de joueurs et les
rumeurs de transfert. Il fournit également des informations sur les
clubs et les compétitions de football.

\textbf{Understat} est un site web qui fournit des statistiques avancées
sur les joueurs et les équipes de football, y compris des données sur
les tirs, les passes et les actions défensives.

Bien que notre sujet soit le championnat anglais, nous pourrons
également être amenés à le comparer aux autres championnats majeurs,
afin de voir si les préjugés sur cette ligue sont fondés ou non.

Étant donné que la librairie que nous utilisons nous donne accès à un
nombre gigantesque de données, nous avons décidé de n'utiliser que les
données nécessaires pour répondre à nos questions de recherche. Cela
nous permettra de nous concentrer sur les informations pertinentes et de
ne pas être submergés par des données inutiles.

Lorsque l'on effectue des requêtes sur une API, il est fréquent d'être
limité dans le nombre de requêtes que l'on peut effectuer dans un
certain laps de temps. Dans ce cas, nous stockerons les données
récupérées dans des fichiers CSV afin de pouvoir les réutiliser
ultérieurement sans avoir à effectuer de nouvelles requêtes et de ne pas
dépendre de l'API en cas de panne ou de changement dans les données
fournies.

\textbf{Présentation des données} Le jeu de données nous permet
d'obtenir des informations à plusieurs échelles :

\textbf{À l'échelle des saisons} On peut retrouver à l'échelle des
saisons des données sur les équipes et les joueurs grâce à la méthode
\texttt{fb\_big5\_advanced\_season\_stats()}, par exemple, pour obtenir
la possession des joueurs en 2021, on peut utiliser la commande
\texttt{fb\_big5\_advanced\_season\_stats(season\_end\_year=2021,stat\_type="possession",team\_or\_player="player")},
qui nous renverra le dataframe des joueurs avec les informations
suivantes :

\begin{itemize}
\item
  \texttt{Squad} : le nom de l'équipe du joueur, donnée nominale
\item
  \texttt{Player} : le nom du joueur, donnée nominale
\item
  \texttt{Nation} : la nationalité du joueur, donnée nominale
\item
  \texttt{Pos} : le poste du joueur, donnée nominale
\item
  \texttt{Age} : l'âge du joueur, donnée quantitative
\item
  \texttt{Born} : la date de naissance du joueur, donnnée nominale
\item
  \texttt{Mins\_per90} : le nombre de minutes jouées par match, donnée
  quantitative
\item
  \texttt{Touches\_Touches} : le nombre de touches de balle par match,
  donnée quantitative
\item
  \texttt{Touches\_Def\_Pen} : le nombre de touches de balle dans la
  surface de réparation adverse par match, donnée quantitative
\item
  \texttt{Succ\_percent\_Dribbles}: le pourcentage de réussite des
  dribbles par match, donnée quantitative
\end{itemize}

Cette liste n'est pas exhaustive, car il y a en réalité 32 colonnes dans
le dataframe, mais cela donne une idée des informations que l'on peut
obtenir et surtout de ce que nous allons avoir besoin pour répondre à
nos questions.

\textbf{À l'échelle des équipes} On peut retrouver à l'échelle des
équipes diverses données grâce à la méthode
\texttt{fb\_team\_match\_log\_stats(team\_urls,\ stat\_type)} qui nous
renverra le dataframe des équipes avec les informations suivantes :

\begin{itemize}
\item
  \texttt{Team} : le nom de l'équipe, donnée nominale
\item
  \texttt{Date} : la date du match, donnée nominale
\item
  \texttt{Time} : l'heure du match, donnée nominale
\item
  \texttt{Comp} : la compétition, donnée nominale
\item
  \texttt{Round} : le tour de la compétition, donnée nominale
\item
  \texttt{Day} : le jour du match, donnée nominale
\item
  \texttt{Venue} : le lieu du match, donnée nominale
\item
  \texttt{Result} : le résultat du match, donnée nominale
\item
  \texttt{Opponent} : l'équipe adverse, donnée nominale
\end{itemize}

Le reste des données dépend du stat\_type que l'on choisit, par exemple,
si on choisit \texttt{stat\_type="passing"}, ou
\texttt{stat\_type="defense"}, on aura des informations sur les passes
ou la défense de l'équipe, respectivement.

\textbf{À l'échelle des joueurs} Enfin, on peut retrouver à l'échelle
des joueurs diverses données grâce à la méthode
\texttt{fb\_player\_season\_stats(player\_url,\ stat\_type)} qui nous
renverra le dataframe des joueurs avec les informations suivantes :

\begin{itemize}
\item
  \texttt{player\_name} : le nom du joueur, donnée nominale
\item
  \texttt{Season} : la saison, de forme AN01-AN02, donnée nominale
\item
  \texttt{Age} : l'âge du joueur, donnée quantitative
\item
  \texttt{Squad} : le nom de l'équipe du joueur, donnée nominale
\item
  \texttt{Country} : le pays du joueur, donnée nominale
\item
  \texttt{Comp} : la compétition, donnée nominale
\item
  \texttt{MP} : le nombre de matchs joués, donnée quantitative
\item
  \texttt{Starts\_Time}: le nombre de matchs joués en tant que
  titulaire, donnée quantitative
\item
  \texttt{Gls} : le nombre de buts marqués, donnée quantitative
\end{itemize}

On peut rajouter une deuxième méthode tm\_player\_bio() qui retourne des
informations supplémentaires.

\begin{itemize}
\item
  \texttt{player\_name} : Nom du joueur, donnée nominale
\item
  \texttt{date\_of\_birth} : Date de naissance, donnée nominale
\item
  \texttt{place\_of\_birth} : Lieu de naissance, donnée nominale
\item
  \texttt{height} : Taille, donnée quantitative
\item
  \texttt{nationality} : Nationalité, donnée nominale
\item
  \texttt{position} : Poste, donnée nominale
\item
  \texttt{strong\_foot} : Pied fort, donnée nominale
\item
  \texttt{current\_club} : Club actuel, donnée nominale
\item
  \texttt{joined} : Date d`arrivée dans le club actuel, donnée nominale
\item
  \texttt{contract\_expires} : Date d`expiration du contrat, donnée
  nominale
\item
  \texttt{date\_of\_last\_contract\_extension} : Date de la dernière
  extension de contrat, donnée nominale
\item
  \texttt{player\_valuation} : Valeur marchande du joueur, donnée
  nominale
\item
  \texttt{max\_player\_valuation} : Valeur marchande maximale du joueur,
  donnée quantitative
\item
  \texttt{max\_player\_valuation\_date} : Date de la valeur marchande
  maximale du joueur, donnée quantitative
\item
  \texttt{URL} : URL du joueur, donnée nominale
\end{itemize}

⚠️ Warning: On pourrait être amené à puiser des informations sur
d'autres méthodes.

\textbf{Blessure} On aura l'historique des joueurs grâce à
tm\_player\_injury\_history(), qui retourne différentes informations :

\begin{itemize}
\item
  \texttt{player\_url} : l'URL du joueur, donnée nominale
\item
  \texttt{season\_injured} : la saison de blessure du joueur, donnée
  nominale
\item
  \texttt{injury} : le type de blessure du joueur, donnée nominale
\item
  \texttt{injured\_since} : la date de début de la blessure du joueur,
  donnée temporelle
\item
  \texttt{injured\_until} : la date de fin de la blessure du joueur,
  donnée temporelle
\item
  \texttt{duration} : la durée de la blessure du joueur, donnée nominale
\item
  \texttt{games\_missed} : le nombre de matchs manqués par le joueur en
  raison de la blessure, donnée nominale
\item
  \texttt{club} : le club du joueur, donnée nominale
\end{itemize}

Une fois de plus, cette liste n'est pas exhaustive, mais cela donne une
idée des informations que l'on peut obtenir et surtout de ce que nous
allons avoir besoin pour répondre à nos questions.

\#Les questions que nous pouvons nous poser

Nous allons à présent vous présenter les différentes questions que nous
nous sommes posées sur le championnat. Pour chacune de ces questions,
nous allons vous expliquer comment nous allons y répondre et quelles
données nous allons utiliser pour cela.

\textbf{Quelles sont les différences entre les 5 premières équipes de
chaque championnats ? (traité par Ewen)}

Pour répondre à cette question, nous allons nous concentrer sur les 5
premières équipes de chaque championnat pour voir s'il y a des
différences significatives entre elles. Nous allons nous intéresser à
des statistiques comme la possession de balle (obtenu avec le
\texttt{stat\_type="possession"}), le nombre de passes réussies (avec
\texttt{stat\_type="passing"}),puis le nombre de tirs et le nombre de
buts marqués (\texttt{stat\_type="attack"}).

Pour représenter et mettre en relation les données, nous allons utiliser
plusieurs graphiques en barres pour comparer les différentes équipes
entre elles sur les différentes statistiques. Nous pourrons également
utiliser des graphiques en nuages de points pour voir s'il y a une
corrélation entre certaines statistiques. Cela nous permettra d'observer
quelles statistiques sont les plus importantes pour se démarquer des
autres équipes, et si ces statistiques sont inchangées d'un championnat
à l'autre.

\textbf{Le nombre de blessures est-il lié au nombre de match joué ou le
championnat joue une plus grosse partie (stéréotype de championnat +
physique que d'autres ) (traité par Ahmed)}

Pour répondre à cette question, nous utiliserons deux requêtes. La
première nous donnera le nombre de matchs joués, éventuellement
accompagné d'un calcul du temps de jeu. Pour des raisons de clarté et de
compréhension, nous sélectionnerons probablement les 100 joueurs les
plus blessés en utilisant la méthode
\texttt{tm\_player\_injury\_history()} et en filtrant sur la somme de la
différence entre injured\_until et injured\_since.

\textbf{Quelles sont les blessures les plus fréquentes pour les joueurs
de Premier League ? (traitée par Ahmed)} Nous récupérerons les blessures
via tm\_player\_injury\_history, qui prend en paramètre un player\_url
de Transfermarkt. Nous regarderons donc les blessures de tous les
joueurs en filtrant sur les joueurs de PL. Un nuage de mots serait bien
pour représenter la répartition. .

\textbf{Quel est le profil de buteur le plus prolifique (avec des
statistiques sur la taille) ?}

Nous examinerons les meilleurs buteurs des championnats et leurs
caractéristiques (physiques, temps de jeu, blessures, etc.) pour trouver
la meilleure corrélation. Nous pourrons également créer une carte
thermique (heatmap) avec PowerBI (ou avec R) sur les origines de ces
joueurs en utilisant la méthode \texttt{tm\_player\_bio()}.

\textbf{Déterminer le profil de l'équipe parfaite (possession par
exemple) (traité par Nassim)}

Nous analyserons les statistiques des meilleures équipes et les
comparerons à celles des équipes moins performantes pour déterminer les
caractéristiques d'une équipe parfaite, telles que la possession de
balle,nombre de buts moyens encaissés\ldots{} On utilisera
\texttt{fb\_team\_match\_log\_stats()} pour recuperer tout les matchs et
faire des moyennes sur leurs matchs .

\textbf{Quelle est la corrélation entre la valeur marchande des joueurs
et leurs postes ? (traité par Amine)}

Nous étudierons la corrélation entre la valeur marchande des joueurs et
différents critères spécifiques à leur poste. Par exemple, nous
examinerons le nombre de buts pour les attaquants et le nombre
d'interceptions pour les défenseurs parmi les joueurs les plus chers.
Pour ce faire, nous utiliserons la méthode mentionnée précédemment dans
la section ``À l'échelle des joueurs''.

\textbf{Quelles sont les différences entre les championnats et les
coupes ? (traité par Nassim)}

Il existe plusieurs différences entre les championnats et les coupes.
Nous pourrions les distinguer en examinant différents aspects tels que
le nombre moyen de buts marqués par match ou les performances moyennes
des joueurs en coupe par rapport au championnat. Par exemple, en
utilisant la méthode \texttt{fb\_player\_scouting\_report()}, nous
pouvons spécifier le championnat comme paramètre et obtenir des retours
standard pour analyser ces différences.

\hypertarget{ruxe9ponses-aux-questions}{%
\section{Réponses aux questions}\label{ruxe9ponses-aux-questions}}

\hypertarget{quelles-sont-les-diffuxe9rences-entre-les-5-premiuxe8res-uxe9quipes-de-chaque-championnats}{%
\subsubsection{Quelles sont les différences entre les 5 premières
équipes de chaque championnats
?}\label{quelles-sont-les-diffuxe9rences-entre-les-5-premiuxe8res-uxe9quipes-de-chaque-championnats}}

(traité par Ewen)

Pour répondre à cette question, nous allons dans un premier temps
effectuer une analyse pour la première League. Nous allons se baser sur
des données concernant la saison 2021-2022, et observer les différentes
statistiques des équipes durant cette saison, afin d'observer
d'éventuels lien entre ces statistiques et la position des équipes dans
le classement.

Avant de commencer, on peut déjà faire plusieurs suppositions :

\begin{itemize}
\item
  On pourrait penser que la meileure équipe aura plus de buts, plus de
  possession et plus de réussite de manière générale dans ses actions,
  ce qui pourrait expliquer le fait qu'elle soit en haut du classement.
\item
  On s'attend aussi à avoir des données relativement proches, et pas une
  équipe qui écrase toute les autres, car bien qu'une équipe soit bas
  dans le classement, les joueurs concernés restent des professionnels.
\end{itemize}

Nous utiliserons des données récupérées sur le sit \textbf{Fbref},
stockées dans un csv. Les données sont les statistiques de chaque
équipes pour chaque saison (Nombre de buts, nombre de touches,
etc\ldots)

\begin{Shaded}
\begin{Highlighting}[]
\FunctionTok{library}\NormalTok{(readr)}
\FunctionTok{library}\NormalTok{(dplyr)}
\FunctionTok{library}\NormalTok{(ggplot2)}
\NormalTok{big5\_data\_by\_team }\OtherTok{\textless{}{-}} \FunctionTok{read\_csv}\NormalTok{(}\StringTok{"./data/dataset/big5\_data\_by\_teams/big5\_data\_by\_team.csv"}\NormalTok{,}\AttributeTok{show\_col\_types =} \ConstantTok{FALSE}\NormalTok{)}
\end{Highlighting}
\end{Shaded}

On va sélectionner les colonnes qui nous intéressent pour notre analyse.
Dans notre cas on va sélectionner les données de 2022 et les données
concernant les 5 meilleures équipes de la première League pour cette
année, à savoir Manchester City, Liverpool, Chelsea, Tottenham, Arsenal.
On va également rajouter une équipe plus basse dans le classement,
Southampton, afin d'éventuellement noter des différences entre le top 5
et des équipes moins bien classées.

Les colonnes concernées sont''Season\_End\_Year'' et ``Squad''

\begin{Shaded}
\begin{Highlighting}[]
\NormalTok{big5\_data\_by\_team }\OtherTok{\textless{}{-}}\NormalTok{ big5\_data\_by\_team }\SpecialCharTok{\%\textgreater{}\%} \FunctionTok{filter}\NormalTok{(Season\_End\_Year }\SpecialCharTok{==} \DecValTok{2022} \SpecialCharTok{\&}\NormalTok{ Squad }\SpecialCharTok{\%in\%} \FunctionTok{c}\NormalTok{(}\StringTok{"Manchester City"}\NormalTok{, }\StringTok{"Liverpool"}\NormalTok{, }\StringTok{"Chelsea"}\NormalTok{, }\StringTok{"Tottenham"}\NormalTok{, }\StringTok{"Arsenal"}\NormalTok{, }\StringTok{"Southampton"}\NormalTok{))}
\end{Highlighting}
\end{Shaded}

Il y a énormément de colonnes à notre disposition, on va donc
sélectionner les colonnes qui nous intéressent pour notre analyse. On va
garder uniquement Squad, Team\_or\_Opponent, Gls, Poss.y,
Touches\_Touches, Succ\_percent\_Dribbles

\begin{Shaded}
\begin{Highlighting}[]
\NormalTok{big5\_data\_by\_team }\OtherTok{\textless{}{-}}\NormalTok{ big5\_data\_by\_team }\SpecialCharTok{\%\textgreater{}\%} \FunctionTok{select}\NormalTok{(Squad, Team\_or\_Opponent, Gls, }\StringTok{"\_percent\_Pressures"}\NormalTok{, Touches\_Touches, Succ\_percent\_Dribbles)}
\end{Highlighting}
\end{Shaded}

On va ensuite renommer les colonnes pour plus de clarté

\begin{Shaded}
\begin{Highlighting}[]
\NormalTok{big5\_data\_by\_team }\OtherTok{\textless{}{-}}\NormalTok{ big5\_data\_by\_team }\SpecialCharTok{\%\textgreater{}\%} \FunctionTok{rename}\NormalTok{(}\AttributeTok{Equipe =}\NormalTok{ Squad, }\AttributeTok{Buts =}\NormalTok{ Gls, }\AttributeTok{Reussite\_pressing =} \StringTok{\textquotesingle{}\_percent\_Pressures\textquotesingle{}}\NormalTok{, }\AttributeTok{Touches =}\NormalTok{ Touches\_Touches, }\AttributeTok{Reussite\_Dribble =}\NormalTok{ Succ\_percent\_Dribbles)}
\end{Highlighting}
\end{Shaded}

Team\_or\_Opponent distingue les match où l'équipe est à domicile ou à
l'extérieur. On va donc regrouper les données par équipe et par match à
domicile ou à l'extérieur.

\begin{Shaded}
\begin{Highlighting}[]
\NormalTok{big5\_data\_by\_team }\OtherTok{\textless{}{-}}\NormalTok{ big5\_data\_by\_team }\SpecialCharTok{\%\textgreater{}\%} \FunctionTok{group\_by}\NormalTok{(Equipe) }\SpecialCharTok{\%\textgreater{}\%} \FunctionTok{summarise}\NormalTok{(}\AttributeTok{Buts =} \FunctionTok{sum}\NormalTok{(Buts), }\AttributeTok{Reussite\_pressing =} \FunctionTok{mean}\NormalTok{(Reussite\_pressing), }\AttributeTok{Touches =} \FunctionTok{sum}\NormalTok{(Touches), }\AttributeTok{Reussite\_Dribble =} \FunctionTok{mean}\NormalTok{(Reussite\_Dribble))}
\end{Highlighting}
\end{Shaded}

Avant de faire des visualisations, on va ordonner les lignes en fonction
des positions des équipes dans le classement de la ligue, à savoir
Manchester City, Liverpool, Chelsea, Tottenham, Arsenal

\begin{Shaded}
\begin{Highlighting}[]
\NormalTok{big5\_data\_by\_team }\OtherTok{\textless{}{-}}\NormalTok{ big5\_data\_by\_team }\SpecialCharTok{\%\textgreater{}\%} \FunctionTok{mutate}\NormalTok{(}\AttributeTok{Equipe =} \FunctionTok{factor}\NormalTok{(Equipe, }\AttributeTok{levels =} \FunctionTok{c}\NormalTok{(}\StringTok{"Manchester City"}\NormalTok{, }\StringTok{"Liverpool"}\NormalTok{, }\StringTok{"Chelsea"}\NormalTok{, }\StringTok{"Tottenham"}\NormalTok{, }\StringTok{"Arsenal"}\NormalTok{, }\StringTok{"Southampton"}\NormalTok{)))}
\end{Highlighting}
\end{Shaded}

On peut ensuite faire une première visualisation, avec un barplot qui
affiche le nombre de but marqué par chaque équipe

\begin{Shaded}
\begin{Highlighting}[]
\FunctionTok{ggplot}\NormalTok{(big5\_data\_by\_team, }\FunctionTok{aes}\NormalTok{(}\AttributeTok{x =}\NormalTok{ Equipe, }\AttributeTok{y =}\NormalTok{ Buts, }\AttributeTok{fill =}\NormalTok{ Equipe)) }\SpecialCharTok{+} \FunctionTok{geom\_bar}\NormalTok{(}\AttributeTok{stat =} \StringTok{"identity"}\NormalTok{) }\SpecialCharTok{+} \FunctionTok{theme\_minimal}\NormalTok{() }\SpecialCharTok{+} \FunctionTok{labs}\NormalTok{(}\AttributeTok{title =} \StringTok{"Nombre de buts marqués par équipe"}\NormalTok{, }\AttributeTok{x =} \StringTok{"Equipe"}\NormalTok{, }\AttributeTok{y =} \StringTok{"Nombre de buts"}\NormalTok{)}
\end{Highlighting}
\end{Shaded}

\includegraphics{rapport_files/figure-latex/unnamed-chunk-7-1.pdf}

Comme on pouvait s'y attendre, Manchester City, Liverpool et Chelsea,
respectivement 1ère, 2ème et 3ème équipe du classement, sont les équipes
qui possède le plus de buts. Le nombre de buts est donc directement lié
à la position de l'équipe dans la ligue. Ce résultat était plutôt
attendu, dans un match de football, le nombre de but est le paramètre
qui va déterminer si l'on gagne ou non, on s'attend donc à ce que la
meilleure équipe en ait le plus.

Mais cela nous amène donc à nous demander si d'autres paramètres, plutôt
axés sur la défense, comme le pourcentage de pressing réussis, ou sur la
possession avec le nombre de touches, ou sur l'attaque avec le
pourcentage de dribble réussis, sont également des caractéristiques
importante pour différencier les équipes entre elles. On va donc
visualiser ces paramètres, de la même manière que pour les buts.

\begin{Shaded}
\begin{Highlighting}[]
\FunctionTok{ggplot}\NormalTok{(big5\_data\_by\_team, }\FunctionTok{aes}\NormalTok{(}\AttributeTok{x =}\NormalTok{ Equipe, }\AttributeTok{y =}\NormalTok{ Reussite\_pressing, }\AttributeTok{fill =}\NormalTok{ Equipe)) }\SpecialCharTok{+} \FunctionTok{geom\_bar}\NormalTok{(}\AttributeTok{stat =} \StringTok{"identity"}\NormalTok{) }\SpecialCharTok{+} \FunctionTok{theme\_minimal}\NormalTok{() }\SpecialCharTok{+} \FunctionTok{labs}\NormalTok{(}\AttributeTok{title =} \StringTok{"Pourcentage de pressing réussis par équipe"}\NormalTok{, }\AttributeTok{x =} \StringTok{"Equipe"}\NormalTok{, }\AttributeTok{y =} \StringTok{"Pourcentage de pressing réussis"}\NormalTok{)}
\end{Highlighting}
\end{Shaded}

\includegraphics{rapport_files/figure-latex/unnamed-chunk-8-1.pdf}

\begin{Shaded}
\begin{Highlighting}[]
\FunctionTok{ggplot}\NormalTok{(big5\_data\_by\_team, }\FunctionTok{aes}\NormalTok{(}\AttributeTok{x =}\NormalTok{ Equipe, }\AttributeTok{y =}\NormalTok{ Touches, }\AttributeTok{fill =}\NormalTok{ Equipe)) }\SpecialCharTok{+} \FunctionTok{geom\_bar}\NormalTok{(}\AttributeTok{stat =} \StringTok{"identity"}\NormalTok{) }\SpecialCharTok{+} \FunctionTok{theme\_minimal}\NormalTok{() }\SpecialCharTok{+} \FunctionTok{labs}\NormalTok{(}\AttributeTok{title =} \StringTok{"Nombre de touches par équipe"}\NormalTok{, }\AttributeTok{x =} \StringTok{"Equipe"}\NormalTok{, }\AttributeTok{y =} \StringTok{"Nombre de touches"}\NormalTok{)}
\end{Highlighting}
\end{Shaded}

\includegraphics{rapport_files/figure-latex/unnamed-chunk-8-2.pdf}

\begin{Shaded}
\begin{Highlighting}[]
\FunctionTok{ggplot}\NormalTok{(big5\_data\_by\_team, }\FunctionTok{aes}\NormalTok{(}\AttributeTok{x =}\NormalTok{ Equipe, }\AttributeTok{y =}\NormalTok{ Reussite\_Dribble, }\AttributeTok{fill =}\NormalTok{ Equipe)) }\SpecialCharTok{+} \FunctionTok{geom\_bar}\NormalTok{(}\AttributeTok{stat =} \StringTok{"identity"}\NormalTok{) }\SpecialCharTok{+} \FunctionTok{theme\_minimal}\NormalTok{() }\SpecialCharTok{+} \FunctionTok{labs}\NormalTok{(}\AttributeTok{title =} \StringTok{"Pourcentage de dribbles réussis par équipe"}\NormalTok{, }\AttributeTok{x =} \StringTok{"Equipe"}\NormalTok{, }\AttributeTok{y =} \StringTok{"Pourcentage de dribbles réussis"}\NormalTok{)}
\end{Highlighting}
\end{Shaded}

\includegraphics{rapport_files/figure-latex/unnamed-chunk-8-3.pdf}

On constate qu'il n'y a finalement que très peu de différence entre les
équipes pour ces paramètres. On peut donc en conclure que ces paramètres
ne sont pas déterminants pour différencier les équipes entre elles. Cela
peut s'expliquer par le fait que les équipes de haut niveau ont des
joueurs de qualité, et que ces paramètres sont donc assez homogènes
entre les équipes de haut niveau. On va ensuite rajouter une équipe
moins bonne dans le classement, pour voir si ces paramètres sont plus
déterminants pour différencier les équipes de bas niveau. On va rajouter
l'équipe de Southampton, qui est 15ème de la ligue, et donc une équipe
de bas niveau. On refait les même étapes que précédemment, en rajoutant
Southampton dans les équipes sélectionnées.

On constate finalement que deux paramètres semblent être lié à la
position des équipes dans la ligue : le nombre de but, et le nombre de
touches. Pour les autres, il est possible que le fait que ce sont des
pourcentages de réussites ne permettent pas d'identifier une
corrélation, on peut éventuellement s'intéresser au nombre de dribbles
et de pressing reussis, et non pas aux pourcentages de réussite.

\includegraphics{rapport_files/figure-latex/unnamed-chunk-9-1.pdf}
\includegraphics{rapport_files/figure-latex/unnamed-chunk-9-2.pdf}

À nouveau, on ne constate aucun lien particulier entre ces paramètres et
la position des équipes dans la ligue. On peut donc en conclure que le
nombre de buts et le nombre de touches sont les paramètres les plus
importants pour différencier les équipes entre elles.

On peut maintenant se demander si pour d'autres championnat, comme la
Ligue 1, le nombre de buts et le nombres de touches sont également les
paramètres les plus importants. On effectue donc la même analyse, mais
avec les équipes Paris-SG, Marseille, Monaco, Rennes, Nice, et Troyes
(15ème du classement). On ne va regarder que le nombre de buts et le
nombre de touches.

\includegraphics{rapport_files/figure-latex/unnamed-chunk-10-1.pdf}
\includegraphics{rapport_files/figure-latex/unnamed-chunk-10-2.pdf}

On constate que le nombre de buts semble être moins déterminant pour
différencier les équipes entre elles en Ligue 1, notamment en étudiant
le cas de Rennes. En effet, si l'on se base uniquement se graphe, on
aurait pu pensé qu'elle se placerait à la deuxième place.

On constate aussi que le nombre de touches est un paramètre plus
important, et que le nombre de touche global semble être légèrement plus
élevé.

Cela peut s'expliquer par le fait que la Ligue 1 est un championnat
moins offensif que la Premier League, et que la possession de balle est
donc un paramètre plus important pour différencier les équipes entre
elles.

\hypertarget{quelles-sont-les-blessures-les-plus-fruxe9quentes-pour-les-joueurs-de-premier-league}{%
\subsubsection{\texorpdfstring{\textbf{Quelles sont les blessures les
plus fréquentes pour les joueurs de Premier League
?}}{Quelles sont les blessures les plus fréquentes pour les joueurs de Premier League ?}}\label{quelles-sont-les-blessures-les-plus-fruxe9quentes-pour-les-joueurs-de-premier-league}}

(traité par Ahmed)

Dans cette analyse, nous chercherons à identifier les types de blessures
les plus courantes et les parties du corps les plus affectées chez les
joueurs professionnels, en mettant particulièrement l'accent sur les
joueurs de Premier League. Nous pourrions nous attendre à ce que les
blessures aux chevilles, aux genoux et aux adducteurs soient parmi les
plus fréquentes.

\textbf{Traitement des données} : Dans le cadre de notre analyse des
blessures des joueurs de Premier League, nous avons collecté des données
à partir de différentes sources. Tout d'abord, nous avons créé une liste
de tous les clubs de Premier League, ``clubs\_list'', contenant 26
clubs.

Ensuite, nous avons importé les données relatives aux joueurs à partir
du fichier CSV ``players.csv'' en utilisant la fonction
readr::read\_csv().

\begin{Shaded}
\begin{Highlighting}[]
\NormalTok{players }\OtherTok{\textless{}{-}}\NormalTok{ readr}\SpecialCharTok{::}\FunctionTok{read\_csv}\NormalTok{(}\StringTok{"dataset/playerslinkALL/players.csv"}\NormalTok{)}
\FunctionTok{colnames}\NormalTok{(players)}
\end{Highlighting}
\end{Shaded}

Pour obtenir les statistiques de blessures de chaque joueur, nous avons
utilisé une fonction personnalisée, tm\_player\_injury\_history(), qui
prend l'URL du profil du joueur sur le site web de Transfermarkt comme
argument. Nous avons itéré sur chaque ligne de notre jeu de données
``players'' en utilisant une boucle for, et pour chaque joueur, nous
avons récupéré ses statistiques de blessures en utilisant cette
fonction.

\begin{Shaded}
\begin{Highlighting}[]
\NormalTok{players\_stats\_list }\OtherTok{\textless{}{-}} \FunctionTok{list}\NormalTok{()}

\ControlFlowTok{for}\NormalTok{ (i }\ControlFlowTok{in} \DecValTok{1}\SpecialCharTok{:}\FunctionTok{nrow}\NormalTok{(players)) \{}
\NormalTok{  player\_url }\OtherTok{\textless{}{-}}\NormalTok{ players}\SpecialCharTok{$}\NormalTok{UrlTmarkt[i]}
\NormalTok{  stat }\OtherTok{\textless{}{-}} \FunctionTok{tm\_player\_injury\_history}\NormalTok{(player\_url)}
  \ControlFlowTok{if}\NormalTok{ (}\FunctionTok{any}\NormalTok{(}\SpecialCharTok{!}\FunctionTok{is.na}\NormalTok{(stat}\SpecialCharTok{$}\NormalTok{club))) \{}
    \FunctionTok{print}\NormalTok{(}\StringTok{"club"}\NormalTok{)}
\NormalTok{    found }\OtherTok{\textless{}{-}} \ConstantTok{FALSE}
\NormalTok{    clubs }\OtherTok{\textless{}{-}} \FunctionTok{strsplit}\NormalTok{(stat}\SpecialCharTok{$}\NormalTok{club, }\StringTok{", "}\NormalTok{)[[}\DecValTok{1}\NormalTok{]]}
    \ControlFlowTok{for}\NormalTok{ (club }\ControlFlowTok{in}\NormalTok{ clubs) \{}
\NormalTok{      found }\OtherTok{\textless{}{-}} \FunctionTok{grepl}\NormalTok{(club, clubs\_list, }\AttributeTok{ignore.case =} \ConstantTok{TRUE}\NormalTok{)}
      \ControlFlowTok{if}\NormalTok{ (}\FunctionTok{any}\NormalTok{(found)) \{}
        \FunctionTok{print}\NormalTok{(}\StringTok{"en PL"}\NormalTok{)}
\NormalTok{        players\_stats\_list[[i]] }\OtherTok{\textless{}{-}}\NormalTok{ stat}
        \ControlFlowTok{break}
\NormalTok{      \} }\ControlFlowTok{else}\NormalTok{ \{}
        \FunctionTok{print}\NormalTok{(}\StringTok{"pas en PL "}\NormalTok{)}
\NormalTok{      \}}
\NormalTok{    \}}
\NormalTok{  \}}
\NormalTok{\}}
\end{Highlighting}
\end{Shaded}

Comme un joueur peut avoir joué pour plusieurs clubs au cours de sa
carrière, nous avons vérifié s'il a joué pour un club de Premier League
en comparant le nom du club dans les statistiques de blessures avec
notre liste de clubs de Premier League. Nous avons utilisé la fonction
strsplit() pour diviser la chaîne de caractères contenant les noms des
clubs en un vecteur, puis nous avons itéré sur chaque club en utilisant
une boucle for. Nous avons utilisé la fonction grepl() pour vérifier si
le nom du club est dans notre liste de clubs de Premier League, en
ignorant la casse. Si le joueur a joué pour un club de Premier League,
nous avons ajouté ses statistiques de blessures à notre liste
``players\_stats\_list''.

\begin{Shaded}
\begin{Highlighting}[]
\NormalTok{players\_stats\_df }\OtherTok{\textless{}{-}} \FunctionTok{do.call}\NormalTok{(rbind, players\_stats\_list)}
\end{Highlighting}
\end{Shaded}

\hypertarget{graphique}{%
\subsubsection{Graphique}\label{graphique}}

Cette visualisation nous montre les blessures les plus récurrentes avec
un double encodage pour représenter la fréquence, la taille et la
couleur. Ce graphique a été réalisé à partir de données recensant 1000
blessures de joueurs ayant évolué en Premier League.

\hypertarget{analyse-des-graphiques}{%
\subsubsection{Analyse des graphiques}\label{analyse-des-graphiques}}

Les résultats confirment nos attentes, avec comme top 3 des blessures
les plus récurrentes : en premier, les adducteurs, suivis des genoux,
puis des chevilles. Cette observation est logique étant donné que le
football de haut niveau exerce une pression particulière sur certains
muscles et articulations. Nous remarquons également l'impact de la crise
du COVID-19, qui a réussi à figurer parmi les blessures les plus
récurrentes, bien que son impact ait été de courte durée dans les
rapports de blessures (avant la suspension du championnat).

\hypertarget{le-nombre-de-blessures-est-il-liuxe9-au-nombre-de-match-jouuxe9-ou-le-championnat-joue-une-plus-grosse-partie-stuxe9ruxe9otype-de-championnat-physique-que-dautres}{%
\subsubsection{\texorpdfstring{\textbf{Le nombre de blessures est-il lié
au nombre de match joué ou le championnat joue une plus grosse partie
(stéréotype de championnat + physique que d'autres
)}}{Le nombre de blessures est-il lié au nombre de match joué ou le championnat joue une plus grosse partie (stéréotype de championnat + physique que d'autres )}}\label{le-nombre-de-blessures-est-il-liuxe9-au-nombre-de-match-jouuxe9-ou-le-championnat-joue-une-plus-grosse-partie-stuxe9ruxe9otype-de-championnat-physique-que-dautres}}

(traité par Ahmed) \textbf{optionnel dans la deadline du 5 mai}

Dans cette analyse, nous nous intéressons à la relation entre le nombre
de jours entre chaque match et le nombre de blessures dans les ligues
majeures de football. Notre question de recherche est de savoir si
certains championnats sont plus sujets aux blessures que d'autres en
raison de leur nature plus physique supposée . Dans l'imaginaire
collectif des fans de football, la Premier League a une réputation de
championnat plus physique que les autres. On pourrait donc s'attendre à
plus de blessures de la part des joueurs de cette ligue.

\hypertarget{graphiques}{%
\subsubsection{Graphiques}\label{graphiques}}

Sur ce graphique, l'axe des abscisses représente le nombre moyen de
jours entre chaque match, tandis que l'axe des ordonnées représente le
nombre de blessures. Nous observons une tendance générale où les
championnats avec moins de jours de repos entre les matchs ont tendance
à avoir un nombre de blessures plus élevé.

Sur ce graphique, l'axe des abscisses représente le nombre de matchs par
équipe en championnat, tandis que l'axe des ordonnées représente le
nombre de blessures. Nous observons une relation linéaire entre le
nombre de matchs et le nombre de blessures, sauf pour la Ligue 1 qui
semble être une exception à cette tendance.

\hypertarget{analyse-des-graphiques-1}{%
\subsubsection{Analyse des graphiques}\label{analyse-des-graphiques-1}}

Les graphiques montrent clairement des tendances. Nous constatons
presque une relation linéaire entre le nombre de matchs et les
blessures, à l'exception de la Ligue 1. Cela confirme l'idée que
certains championnats peuvent être plus sujets aux blessures en raison
de leur intensité de matchs.

En ce qui concerne la relation entre le délai entre chaque match et les
blessures, nous observons que les délais sont souvent très proches, mais
nous remarquons deux extrêmes. La Premier League a le plus grand nombre
de blessures avec le moins de temps de repos, tandis que la Ligue 1 a le
nombre de blessures le plus bas avec le temps de repos le plus long.
Cela suggère que plus une équipe a de temps de repos, moins elle est
exposée aux blessures.

Cependant, la Bundesliga se distingue en ayant un bon temps de repos
mais un nombre élevé de blessures. Cela soulève la question de savoir
quels autres facteurs influencent le nombre de blessures, en dehors du
nombre de matchs. Il est important de prendre en compte que le nombre de
matchs peut varier selon les équipes au sein d'un même championnat, en
fonction de leur participation dans les compétitions de coupe.

En conclusion, ces observations soulignent l'importance du repos entre
les matchs dans la prévention des blessures, mais elles soulèvent
également d'autres questions sur les facteurs influençant le nombre de
blessures dans le football professionnel car le repos n'étant pas
l'indice absolu . On pourrait donc se demander quels sont ces autres
indices ?

\end{document}
