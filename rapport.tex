% Options for packages loaded elsewhere
\PassOptionsToPackage{unicode}{hyperref}
\PassOptionsToPackage{hyphens}{url}
%
\documentclass[
]{article}
\usepackage{amsmath,amssymb}
\usepackage{iftex}
\ifPDFTeX
  \usepackage[T1]{fontenc}
  \usepackage[utf8]{inputenc}
  \usepackage{textcomp} % provide euro and other symbols
\else % if luatex or xetex
  \usepackage{unicode-math} % this also loads fontspec
  \defaultfontfeatures{Scale=MatchLowercase}
  \defaultfontfeatures[\rmfamily]{Ligatures=TeX,Scale=1}
\fi
\usepackage{lmodern}
\ifPDFTeX\else
  % xetex/luatex font selection
\fi
% Use upquote if available, for straight quotes in verbatim environments
\IfFileExists{upquote.sty}{\usepackage{upquote}}{}
\IfFileExists{microtype.sty}{% use microtype if available
  \usepackage[]{microtype}
  \UseMicrotypeSet[protrusion]{basicmath} % disable protrusion for tt fonts
}{}
\makeatletter
\@ifundefined{KOMAClassName}{% if non-KOMA class
  \IfFileExists{parskip.sty}{%
    \usepackage{parskip}
  }{% else
    \setlength{\parindent}{0pt}
    \setlength{\parskip}{6pt plus 2pt minus 1pt}}
}{% if KOMA class
  \KOMAoptions{parskip=half}}
\makeatother
\usepackage{xcolor}
\usepackage[margin=1in]{geometry}
\usepackage{color}
\usepackage{fancyvrb}
\newcommand{\VerbBar}{|}
\newcommand{\VERB}{\Verb[commandchars=\\\{\}]}
\DefineVerbatimEnvironment{Highlighting}{Verbatim}{commandchars=\\\{\}}
% Add ',fontsize=\small' for more characters per line
\usepackage{framed}
\definecolor{shadecolor}{RGB}{248,248,248}
\newenvironment{Shaded}{\begin{snugshade}}{\end{snugshade}}
\newcommand{\AlertTok}[1]{\textcolor[rgb]{0.94,0.16,0.16}{#1}}
\newcommand{\AnnotationTok}[1]{\textcolor[rgb]{0.56,0.35,0.01}{\textbf{\textit{#1}}}}
\newcommand{\AttributeTok}[1]{\textcolor[rgb]{0.13,0.29,0.53}{#1}}
\newcommand{\BaseNTok}[1]{\textcolor[rgb]{0.00,0.00,0.81}{#1}}
\newcommand{\BuiltInTok}[1]{#1}
\newcommand{\CharTok}[1]{\textcolor[rgb]{0.31,0.60,0.02}{#1}}
\newcommand{\CommentTok}[1]{\textcolor[rgb]{0.56,0.35,0.01}{\textit{#1}}}
\newcommand{\CommentVarTok}[1]{\textcolor[rgb]{0.56,0.35,0.01}{\textbf{\textit{#1}}}}
\newcommand{\ConstantTok}[1]{\textcolor[rgb]{0.56,0.35,0.01}{#1}}
\newcommand{\ControlFlowTok}[1]{\textcolor[rgb]{0.13,0.29,0.53}{\textbf{#1}}}
\newcommand{\DataTypeTok}[1]{\textcolor[rgb]{0.13,0.29,0.53}{#1}}
\newcommand{\DecValTok}[1]{\textcolor[rgb]{0.00,0.00,0.81}{#1}}
\newcommand{\DocumentationTok}[1]{\textcolor[rgb]{0.56,0.35,0.01}{\textbf{\textit{#1}}}}
\newcommand{\ErrorTok}[1]{\textcolor[rgb]{0.64,0.00,0.00}{\textbf{#1}}}
\newcommand{\ExtensionTok}[1]{#1}
\newcommand{\FloatTok}[1]{\textcolor[rgb]{0.00,0.00,0.81}{#1}}
\newcommand{\FunctionTok}[1]{\textcolor[rgb]{0.13,0.29,0.53}{\textbf{#1}}}
\newcommand{\ImportTok}[1]{#1}
\newcommand{\InformationTok}[1]{\textcolor[rgb]{0.56,0.35,0.01}{\textbf{\textit{#1}}}}
\newcommand{\KeywordTok}[1]{\textcolor[rgb]{0.13,0.29,0.53}{\textbf{#1}}}
\newcommand{\NormalTok}[1]{#1}
\newcommand{\OperatorTok}[1]{\textcolor[rgb]{0.81,0.36,0.00}{\textbf{#1}}}
\newcommand{\OtherTok}[1]{\textcolor[rgb]{0.56,0.35,0.01}{#1}}
\newcommand{\PreprocessorTok}[1]{\textcolor[rgb]{0.56,0.35,0.01}{\textit{#1}}}
\newcommand{\RegionMarkerTok}[1]{#1}
\newcommand{\SpecialCharTok}[1]{\textcolor[rgb]{0.81,0.36,0.00}{\textbf{#1}}}
\newcommand{\SpecialStringTok}[1]{\textcolor[rgb]{0.31,0.60,0.02}{#1}}
\newcommand{\StringTok}[1]{\textcolor[rgb]{0.31,0.60,0.02}{#1}}
\newcommand{\VariableTok}[1]{\textcolor[rgb]{0.00,0.00,0.00}{#1}}
\newcommand{\VerbatimStringTok}[1]{\textcolor[rgb]{0.31,0.60,0.02}{#1}}
\newcommand{\WarningTok}[1]{\textcolor[rgb]{0.56,0.35,0.01}{\textbf{\textit{#1}}}}
\usepackage{graphicx}
\makeatletter
\def\maxwidth{\ifdim\Gin@nat@width>\linewidth\linewidth\else\Gin@nat@width\fi}
\def\maxheight{\ifdim\Gin@nat@height>\textheight\textheight\else\Gin@nat@height\fi}
\makeatother
% Scale images if necessary, so that they will not overflow the page
% margins by default, and it is still possible to overwrite the defaults
% using explicit options in \includegraphics[width, height, ...]{}
\setkeys{Gin}{width=\maxwidth,height=\maxheight,keepaspectratio}
% Set default figure placement to htbp
\makeatletter
\def\fps@figure{htbp}
\makeatother
\setlength{\emergencystretch}{3em} % prevent overfull lines
\providecommand{\tightlist}{%
  \setlength{\itemsep}{0pt}\setlength{\parskip}{0pt}}
\setcounter{secnumdepth}{-\maxdimen} % remove section numbering
\ifLuaTeX
  \usepackage{selnolig}  % disable illegal ligatures
\fi
\usepackage{bookmark}
\IfFileExists{xurl.sty}{\usepackage{xurl}}{} % add URL line breaks if available
\urlstyle{same}
\hypersetup{
  pdftitle={Analyse de la Premier League},
  pdfauthor={Groupe Fifa Street},
  hidelinks,
  pdfcreator={LaTeX via pandoc}}

\title{Analyse de la Premier League}
\author{Groupe Fifa Street}
\date{}

\begin{document}
\maketitle

\section{Introduction}\label{introduction}

Nous avons décidé d'orienter notre projet sur la Premier League, qui est
la première division de football en Angleterre. La Premier League fait
partie des 5 grands championnats avec la Ligue 1 (France), la Liga
(Espagne), la Bundesliga (Allemagne) et la Serie A (Italie).

Nous allons utiliser une librairie qui contient une multitude de
fonctions qui requièrent, en fonction des requêtes, une des 3 API des
sites suivants : FBref, Transfermarkt et Understat. Le lien qui décrit
la librairie et toutes ses fonctionnalités est le suivant :
\href{https://jaseziv.github.io/worldfootballR/articles/extract-fbref-data.html}{worldfootballR}

Les sources sont tirées de FBref, Transfermarkt et Understat.

\textbf{FBref} est un site web qui fournit des statistiques détaillées
sur les joueurs et les équipes de football, y compris des données sur
les passes, les tirs et les actions défensives.

\textbf{Transfermarkt} est un site web qui se concentre sur les
transferts de joueurs de football, les évaluations de joueurs et les
rumeurs de transfert. Il fournit également des informations sur les
clubs et les compétitions de football.

\textbf{Understat} est un site web qui fournit des statistiques avancées
sur les joueurs et les équipes de football, y compris des données sur
les tirs, les passes et les actions défensives.

Bien que notre sujet soit le championnat anglais, nous pourrons
également être amenés à le comparer aux autres championnats majeurs,
afin de voir si les préjugés sur cette ligue sont fondés ou non.

Étant donné que la librairie que nous utilisons nous donne accès à un
nombre gigantesque de données, nous avons décidé de n'utiliser que les
données nécessaires pour répondre à nos questions de recherche. Cela
nous permettra de nous concentrer sur les informations pertinentes et de
ne pas être submergés par des données inutiles.

Lorsque l'on effectue des requêtes sur une API, il est fréquent d'être
limité dans le nombre de requêtes que l'on peut effectuer dans un
certain laps de temps. Dans ce cas, nous stockerons les données
récupérées dans des fichiers CSV afin de pouvoir les réutiliser
ultérieurement sans avoir à effectuer de nouvelles requêtes et de ne pas
dépendre de l'API en cas de panne ou de changement dans les données
fournies.

\textbf{Présentation des données} Le jeu de données nous permet
d'obtenir des informations à plusieurs échelles :

\textbf{À l'échelle des saisons} On peut retrouver à l'échelle des
saisons des données sur les équipes et les joueurs grâce à la méthode
\texttt{fb\_big5\_advanced\_season\_stats()}, par exemple, pour obtenir
la possession des joueurs en 2021, on peut utiliser la commande
\texttt{fb\_big5\_advanced\_season\_stats(season\_end\_year=2021,stat\_type="possession",team\_or\_player="player")},
qui nous renverra le dataframe des joueurs avec les informations
suivantes :

\begin{itemize}
\item
  \texttt{Squad} : le nom de l'équipe du joueur, donnée nominale
\item
  \texttt{Player} : le nom du joueur, donnée nominale
\item
  \texttt{Nation} : la nationalité du joueur, donnée nominale
\item
  \texttt{Pos} : le poste du joueur, donnée nominale
\item
  \texttt{Age} : l'âge du joueur, donnée quantitative
\item
  \texttt{Born} : la date de naissance du joueur, donnnée nominale
\item
  \texttt{Mins\_per90} : le nombre de minutes jouées par match, donnée
  quantitative
\item
  \texttt{Touches\_Touches} : le nombre de touches de balle par match,
  donnée quantitative
\item
  \texttt{Touches\_Def\_Pen} : le nombre de touches de balle dans la
  surface de réparation adverse par match, donnée quantitative
\item
  \texttt{Succ\_percent\_Dribbles}: le pourcentage de réussite des
  dribbles par match, donnée quantitative
\end{itemize}

Cette liste n'est pas exhaustive, car il y a en réalité 32 colonnes dans
le dataframe, mais cela donne une idée des informations que l'on peut
obtenir et surtout de ce que nous allons avoir besoin pour répondre à
nos questions.

\textbf{À l'échelle des équipes} On peut retrouver à l'échelle des
équipes diverses données grâce à la méthode
\texttt{fb\_team\_match\_log\_stats(team\_urls,\ stat\_type)} qui nous
renverra le dataframe des équipes avec les informations suivantes :

\begin{itemize}
\item
  \texttt{Team} : le nom de l'équipe, donnée nominale
\item
  \texttt{Date} : la date du match, donnée nominale
\item
  \texttt{Time} : l'heure du match, donnée nominale
\item
  \texttt{Comp} : la compétition, donnée nominale
\item
  \texttt{Round} : le tour de la compétition, donnée nominale
\item
  \texttt{Day} : le jour du match, donnée nominale
\item
  \texttt{Venue} : le lieu du match, donnée nominale
\item
  \texttt{Result} : le résultat du match, donnée nominale
\item
  \texttt{Opponent} : l'équipe adverse, donnée nominale
\end{itemize}

Le reste des données dépend du stat\_type que l'on choisit, par exemple,
si on choisit \texttt{stat\_type="passing"}, ou
\texttt{stat\_type="defense"}, on aura des informations sur les passes
ou la défense de l'équipe, respectivement.

\textbf{À l'échelle des joueurs} Enfin, on peut retrouver à l'échelle
des joueurs diverses données grâce à la méthode
\texttt{fb\_player\_season\_stats(player\_url,\ stat\_type)} qui nous
renverra le dataframe des joueurs avec les informations suivantes :

\begin{itemize}
\item
  \texttt{player\_name} : le nom du joueur, donnée nominale
\item
  \texttt{Season} : la saison, de forme AN01-AN02, donnée nominale
\item
  \texttt{Age} : l'âge du joueur, donnée quantitative
\item
  \texttt{Squad} : le nom de l'équipe du joueur, donnée nominale
\item
  \texttt{Country} : le pays du joueur, donnée nominale
\item
  \texttt{Comp} : la compétition, donnée nominale
\item
  \texttt{MP} : le nombre de matchs joués, donnée quantitative
\item
  \texttt{Starts\_Time}: le nombre de matchs joués en tant que
  titulaire, donnée quantitative
\item
  \texttt{Gls} : le nombre de buts marqués, donnée quantitative
\end{itemize}

On peut rajouter une deuxième méthode tm\_player\_bio() qui retourne des
informations supplémentaires.

\begin{itemize}
\item
  \texttt{player\_name} : Nom du joueur, donnée nominale
\item
  \texttt{date\_of\_birth} : Date de naissance, donnée nominale
\item
  \texttt{place\_of\_birth} : Lieu de naissance, donnée nominale
\item
  \texttt{height} : Taille, donnée quantitative
\item
  \texttt{nationality} : Nationalité, donnée nominale
\item
  \texttt{position} : Poste, donnée nominale
\item
  \texttt{strong\_foot} : Pied fort, donnée nominale
\item
  \texttt{current\_club} : Club actuel, donnée nominale
\item
  \texttt{joined} : Date d`arrivée dans le club actuel, donnée nominale
\item
  \texttt{contract\_expires} : Date d`expiration du contrat, donnée
  nominale
\item
  \texttt{date\_of\_last\_contract\_extension} : Date de la dernière
  extension de contrat, donnée nominale
\item
  \texttt{player\_valuation} : Valeur marchande du joueur, donnée
  nominale
\item
  \texttt{max\_player\_valuation} : Valeur marchande maximale du joueur,
  donnée quantitative
\item
  \texttt{max\_player\_valuation\_date} : Date de la valeur marchande
  maximale du joueur, donnée quantitative
\item
  \texttt{URL} : URL du joueur, donnée nominale
\end{itemize}

⚠️ Warning: On pourrait être amené à puiser des informations sur
d'autres méthodes.

\textbf{Blessure} On aura l'historique des joueurs grâce à
tm\_player\_injury\_history(), qui retourne différentes informations :

\begin{itemize}
\item
  \texttt{player\_url} : l'URL du joueur, donnée nominale
\item
  \texttt{season\_injured} : la saison de blessure du joueur, donnée
  nominale
\item
  \texttt{injury} : le type de blessure du joueur, donnée nominale
\item
  \texttt{injured\_since} : la date de début de la blessure du joueur,
  donnée temporelle
\item
  \texttt{injured\_until} : la date de fin de la blessure du joueur,
  donnée temporelle
\item
  \texttt{duration} : la durée de la blessure du joueur, donnée nominale
\item
  \texttt{games\_missed} : le nombre de matchs manqués par le joueur en
  raison de la blessure, donnée nominale
\item
  \texttt{club} : le club du joueur, donnée nominale
\end{itemize}

Une fois de plus, cette liste n'est pas exhaustive, mais cela donne une
idée des informations que l'on peut obtenir et surtout de ce que nous
allons avoir besoin pour répondre à nos questions.

\#Les questions que nous pouvons nous poser

Nous allons à présent vous présenter les différentes questions que nous
nous sommes posées sur le championnat. Pour chacune de ces questions,
nous allons vous expliquer comment nous allons y répondre et quelles
données nous allons utiliser pour cela.

\textbf{Quelles sont les différences entre les 5 premières équipes de
chaque championnats ? (traité par Ewen)}

Pour répondre à cette question, nous allons nous concentrer sur les 5
premières équipes de chaque championnat pour voir s'il y a des
différences significatives entre elles. Nous allons nous intéresser à
des statistiques comme la possession de balle (obtenu avec le
\texttt{stat\_type="possession"}), le nombre de passes réussies (avec
\texttt{stat\_type="passing"}),puis le nombre de tirs et le nombre de
buts marqués (\texttt{stat\_type="attack"}).

Pour représenter et mettre en relation les données, nous allons utiliser
plusieurs graphiques en barres pour comparer les différentes équipes
entre elles sur les différentes statistiques. Nous pourrons également
utiliser des graphiques en nuages de points pour voir s'il y a une
corrélation entre certaines statistiques. Cela nous permettra d'observer
quelles statistiques sont les plus importantes pour se démarquer des
autres équipes, et si ces statistiques sont inchangées d'un championnat
à l'autre.

\textbf{Le nombre de blessures est-il lié au nombre de match joué ou le
championnat joue une plus grosse partie (stéréotype de championnat +
physique que d'autres ) (traité par Ahmed)}

Pour répondre à cette question, nous utiliserons deux requêtes. La
première nous donnera le nombre de matchs joués, éventuellement
accompagné d'un calcul du temps de jeu. Pour des raisons de clarté et de
compréhension, nous sélectionnerons probablement les 100 joueurs les
plus blessés en utilisant la méthode
\texttt{tm\_player\_injury\_history()} et en filtrant sur la somme de la
différence entre injured\_until et injured\_since.

\textbf{Quelles sont les blessures les plus fréquentes pour les joueurs
de Premier League ? (traitée par Ahmed)} Nous récupérerons les blessures
via tm\_player\_injury\_history, qui prend en paramètre un player\_url
de Transfermarkt. Nous regarderons donc les blessures de tous les
joueurs en filtrant sur les joueurs de PL. Un nuage de mots serait bien
pour représenter la répartition. .

\textbf{Quel est le profil de buteur le plus prolifique (avec des
statistiques sur la taille) ?}

Nous examinerons les meilleurs buteurs des championnats et leurs
caractéristiques (physiques, temps de jeu, blessures, etc.) pour trouver
la meilleure corrélation. Nous pourrons également créer une carte
thermique (heatmap) avec PowerBI (ou avec R) sur les origines de ces
joueurs en utilisant la méthode \texttt{tm\_player\_bio()}.

\textbf{Déterminer le profil de l'équipe parfaite (possession par
exemple) (traité par Nassim)}

Nous analyserons les statistiques des meilleures équipes et les
comparerons à celles des équipes moins performantes pour déterminer les
caractéristiques d'une équipe parfaite, telles que la possession de
balle,nombre de buts moyens encaissés\ldots{} On utilisera
\texttt{fb\_team\_match\_log\_stats()} pour recuperer tout les matchs et
faire des moyennes sur leurs matchs .

\textbf{Quelle est la corrélation entre la valeur marchande des joueurs
et leurs postes ? (traité par Amine)}

Nous étudierons la corrélation entre la valeur marchande des joueurs et
différents critères spécifiques à leur poste. Par exemple, nous
examinerons le nombre de buts pour les attaquants et le nombre
d'interceptions pour les défenseurs parmi les joueurs les plus chers.
Pour ce faire, nous utiliserons la méthode mentionnée précédemment dans
la section ``À l'échelle des joueurs''.

\textbf{Quelles sont les différences entre les championnats et les
coupes ? (traité par Nassim)}

Il existe plusieurs différences entre les championnats et les coupes.
Nous pourrions les distinguer en examinant différents aspects tels que
le nombre moyen de buts marqués par match ou les performances moyennes
des joueurs en coupe par rapport au championnat. Par exemple, en
utilisant la méthode \texttt{fb\_player\_scouting\_report()}, nous
pouvons spécifier le championnat comme paramètre et obtenir des retours
standard pour analyser ces différences.

\section{Réponses aux questions}\label{ruxe9ponses-aux-questions}

\subsubsection{Quelles sont les différences entre les 5 premières
équipes de chaque championnats
?}\label{quelles-sont-les-diffuxe9rences-entre-les-5-premiuxe8res-uxe9quipes-de-chaque-championnats}

(traité par Ewen)

Pour répondre à cette question, nous allons dans un premier temps
effectuer une analyse pour la première League. Nous allons se baser sur
des données concernant la saison 2021-2022, et observer les différentes
statistiques des équipes durant cette saison, afin d'observer
d'éventuels lien entre ces statistiques et la position des équipes dans
le classement.

Avant de commencer, on peut déjà faire plusieurs suppositions :

\begin{itemize}
\item
  On pourrait penser que la meileure équipe aura plus de buts, plus de
  possession et plus de réussite de manière générale dans ses actions,
  ce qui pourrait expliquer le fait qu'elle soit en haut du classement.
\item
  On s'attend aussi à avoir des données relativement proches, et pas une
  équipe qui écrase toute les autres, car bien qu'une équipe soit bas
  dans le classement, les joueurs concernés restent des professionnels.
\end{itemize}

Nous utiliserons des données récupérées sur le site \textbf{Fbref},
stockées dans un csv. Les données sont les statistiques de chaque
équipes pour chaque saison (Nombre de buts, nombre de touches,
etc\ldots)

\begin{Shaded}
\begin{Highlighting}[]
\FunctionTok{library}\NormalTok{(readr)}
\FunctionTok{library}\NormalTok{(dplyr)}
\FunctionTok{library}\NormalTok{(ggplot2)}
\NormalTok{big5\_data\_by\_team }\OtherTok{\textless{}{-}} \FunctionTok{read\_csv}\NormalTok{(}\StringTok{"./data/dataset/big5\_data\_by\_teams/big5\_data\_by\_team.csv"}\NormalTok{,}\AttributeTok{show\_col\_types =} \ConstantTok{FALSE}\NormalTok{)}
\end{Highlighting}
\end{Shaded}

On va sélectionner les colonnes qui nous intéressent pour notre analyse.
Dans notre cas on va sélectionner les données de 2022 et les données
concernant les 5 meilleures équipes de la première League pour cette
année, à savoir Manchester City, Liverpool, Chelsea, Tottenham, Arsenal.
On va également rajouter une équipe plus basse dans le classement,
Southampton, ainsi qu'une équipe moyenne, Newcastle afin
d'éventuellement noter des différences entre le top 5 et des équipes
moins bien classés.

Les colonnes concernées sont''Season\_End\_Year'' et ``Squad''

\begin{Shaded}
\begin{Highlighting}[]
\NormalTok{big5\_data\_by\_team }\OtherTok{\textless{}{-}}\NormalTok{ big5\_data\_by\_team }\SpecialCharTok{\%\textgreater{}\%} \FunctionTok{filter}\NormalTok{(Season\_End\_Year }\SpecialCharTok{==} \DecValTok{2022} \SpecialCharTok{\&}\NormalTok{ Squad }\SpecialCharTok{\%in\%} \FunctionTok{c}\NormalTok{(}\StringTok{"Manchester City"}\NormalTok{, }\StringTok{"Liverpool"}\NormalTok{, }\StringTok{"Chelsea"}\NormalTok{, }\StringTok{"Tottenham"}\NormalTok{, }\StringTok{"Arsenal"}\NormalTok{, }\StringTok{"Newcastle Utd"}\NormalTok{, }\StringTok{"Southampton"}\NormalTok{))}
\end{Highlighting}
\end{Shaded}

Il y a énormément de colonnes à notre disposition, on va donc
sélectionner les colonnes qui nous intéressent pour notre analyse. On va
garder uniquement Squad, Team\_or\_Opponent, Gls, Poss.y,
Touches\_Touches, Succ\_percent\_Dribbles

\begin{Shaded}
\begin{Highlighting}[]
\NormalTok{big5\_data\_by\_team }\OtherTok{\textless{}{-}}\NormalTok{ big5\_data\_by\_team }\SpecialCharTok{\%\textgreater{}\%} \FunctionTok{select}\NormalTok{(Squad, Team\_or\_Opponent, Gls, }\StringTok{"\_percent\_Pressures"}\NormalTok{, Touches\_Touches, Succ\_percent\_Dribbles)}
\end{Highlighting}
\end{Shaded}

On va ensuite renommer les colonnes pour plus de clarté

\begin{Shaded}
\begin{Highlighting}[]
\NormalTok{big5\_data\_by\_team }\OtherTok{\textless{}{-}}\NormalTok{ big5\_data\_by\_team }\SpecialCharTok{\%\textgreater{}\%} \FunctionTok{rename}\NormalTok{(}\AttributeTok{Equipe =}\NormalTok{ Squad, }\AttributeTok{Buts =}\NormalTok{ Gls, }\AttributeTok{Reussite\_pressing =} \StringTok{\textquotesingle{}\_percent\_Pressures\textquotesingle{}}\NormalTok{, }\AttributeTok{Touches =}\NormalTok{ Touches\_Touches, }\AttributeTok{Reussite\_Dribble =}\NormalTok{ Succ\_percent\_Dribbles)}
\end{Highlighting}
\end{Shaded}

Team\_or\_Opponent distingue les match où l'équipe est à domicile ou à
l'extérieur. On va donc regrouper les données par équipe et par match à
domicile ou à l'extérieur.

\begin{Shaded}
\begin{Highlighting}[]
\NormalTok{big5\_data\_by\_team }\OtherTok{\textless{}{-}}\NormalTok{ big5\_data\_by\_team }\SpecialCharTok{\%\textgreater{}\%} \FunctionTok{group\_by}\NormalTok{(Equipe) }\SpecialCharTok{\%\textgreater{}\%} \FunctionTok{summarise}\NormalTok{(}\AttributeTok{Buts =} \FunctionTok{sum}\NormalTok{(Buts), }\AttributeTok{Reussite\_pressing =} \FunctionTok{mean}\NormalTok{(Reussite\_pressing), }\AttributeTok{Touches =} \FunctionTok{mean}\NormalTok{(Touches), }\AttributeTok{Reussite\_Dribble =} \FunctionTok{mean}\NormalTok{(Reussite\_Dribble))}
\end{Highlighting}
\end{Shaded}

Avant de faire des visualisations, on va ordonner les lignes en fonction
des positions des 5 premières équipes dans le classement de la ligue, à
savoir Manchester City, Liverpool, Chelsea, Tottenham, Arsenal On va
ensuite rajouter une équipe moins bonne dans le classement, pour voir si
ces paramètres sont plus déterminants pour différencier les équipes de
bas niveau. On va rajouter l'équipe de Southampton, qui est 15ème de la
ligue, et donc une équipe de bas niveau. On va également rajouter
l'équipe de Newcastle, qui est 11ème de la ligue, et donc une équipe de
milieu de tableau.

\begin{Shaded}
\begin{Highlighting}[]
\NormalTok{big5\_data\_by\_team }\OtherTok{\textless{}{-}}\NormalTok{ big5\_data\_by\_team }\SpecialCharTok{\%\textgreater{}\%} \FunctionTok{mutate}\NormalTok{(}\AttributeTok{Equipe =} \FunctionTok{factor}\NormalTok{(Equipe, }\AttributeTok{levels =} \FunctionTok{c}\NormalTok{(}\StringTok{"Manchester City"}\NormalTok{, }\StringTok{"Liverpool"}\NormalTok{, }\StringTok{"Chelsea"}\NormalTok{, }\StringTok{"Tottenham"}\NormalTok{, }\StringTok{"Arsenal"}\NormalTok{,}\StringTok{"Newcastle Utd"}\NormalTok{, }\StringTok{"Southampton"}\NormalTok{)))}
\end{Highlighting}
\end{Shaded}

On peut ensuite faire une première visualisation, avec un barplot qui
affiche le nombre de but marqué par chaque équipe.

\begin{Shaded}
\begin{Highlighting}[]
\NormalTok{big5\_data\_by\_team }\OtherTok{\textless{}{-}}\NormalTok{ big5\_data\_by\_team }\SpecialCharTok{\%\textgreater{}\%}
  \FunctionTok{mutate}\NormalTok{(}\AttributeTok{Equipe =} \FunctionTok{recode}\NormalTok{(Equipe, }\StringTok{"Manchester City"} \OtherTok{=} \StringTok{"Manchester"}\NormalTok{,}
                                   \StringTok{"Newcastle Utd"} \OtherTok{=}\StringTok{"Newcastle"}\NormalTok{))}
\FunctionTok{ggplot}\NormalTok{(big5\_data\_by\_team, }\FunctionTok{aes}\NormalTok{(}\AttributeTok{x =}\NormalTok{ Equipe, }\AttributeTok{y =}\NormalTok{ Buts, }\AttributeTok{fill =}\NormalTok{ Equipe)) }\SpecialCharTok{+}
  \FunctionTok{geom\_bar}\NormalTok{(}\AttributeTok{stat =} \StringTok{"identity"}\NormalTok{, }\AttributeTok{width =} \FloatTok{0.6}\NormalTok{) }\SpecialCharTok{+}
  \FunctionTok{theme\_minimal}\NormalTok{() }\SpecialCharTok{+}
  \FunctionTok{labs}\NormalTok{(}\AttributeTok{title =} \StringTok{"Nombre de buts marqués par équipe"}\NormalTok{, }\AttributeTok{x =} \StringTok{"Equipe"}\NormalTok{, }\AttributeTok{y =} \StringTok{"Nombre de buts"}\NormalTok{)}\SpecialCharTok{+}
  \FunctionTok{coord\_cartesian}\NormalTok{(}\AttributeTok{ylim =} \FunctionTok{c}\NormalTok{(}\DecValTok{75}\NormalTok{,}\DecValTok{130}\NormalTok{))}
\end{Highlighting}
\end{Shaded}

\includegraphics{rapport_files/figure-latex/unnamed-chunk-7-1.pdf} Comme
on pouvait s'y attendre, Manchester City, Liverpool et Chelsea,
respectivement 1ère, 2ème et 3ème équipe du classement, sont les équipes
qui possède le plus de buts. Le nombre de buts est donc directement lié
à la position de l'équipe dans la ligue. Ce résultat était plutôt
attendu, dans un match de football, le nombre de but est le paramètre
qui va déterminer si l'on gagne ou non, on s'attend donc à ce que la
meilleure équipe en ait le plus. À noter cependant qu'ici, le graphique
commence à partir de 75 buts, cela signifie que dans les fait, aucune
équipe écrase les autres, ce qui est logique dans un championnant d'une
telle envergure.

Mais cela nous amène donc à nous demander si d'autres paramètres, plutôt
axés sur la défense, comme le pourcentage de pressing réussis, ou sur la
possession avec le nombre de touches, ou sur l'attaque avec le
pourcentage de dribble réussis, sont également des caractéristiques
importante pour différencier les équipes entre elles. On va donc
visualiser ces paramètres, de la même manière que pour les buts. Cette
fois-ci, nous allons prendre toute les équipes du championnat. Pour le
nobmre de touches, nous commencerons à 35000 pour mieux identifier
d'éventuelles différences

\begin{Shaded}
\begin{Highlighting}[]
\FunctionTok{ggplot}\NormalTok{(big5\_data\_by\_team, }\FunctionTok{aes}\NormalTok{(}\AttributeTok{x =}\NormalTok{ Equipe, }\AttributeTok{y =}\NormalTok{ Reussite\_pressing, }\AttributeTok{fill =}\NormalTok{ Equipe)) }\SpecialCharTok{+} \FunctionTok{geom\_bar}\NormalTok{(}\AttributeTok{stat =} \StringTok{"identity"}\NormalTok{) }\SpecialCharTok{+} \FunctionTok{theme\_minimal}\NormalTok{() }\SpecialCharTok{+} \FunctionTok{labs}\NormalTok{(}\AttributeTok{title =} \StringTok{"Pourcentage de pressing réussis par équipe"}\NormalTok{, }\AttributeTok{x =} \StringTok{"Equipe"}\NormalTok{, }\AttributeTok{y =} \StringTok{"Pourcentage de pressing réussis"}\NormalTok{)}\SpecialCharTok{+}  \FunctionTok{theme}\NormalTok{(}\AttributeTok{axis.text.x =} \FunctionTok{element\_text}\NormalTok{(}\AttributeTok{angle =} \DecValTok{90}\NormalTok{, }\AttributeTok{vjust =} \FloatTok{0.5}\NormalTok{, }\AttributeTok{hjust =} \DecValTok{1}\NormalTok{))}\SpecialCharTok{+}
  \FunctionTok{guides}\NormalTok{(}\AttributeTok{fill =} \FunctionTok{guide\_legend}\NormalTok{(}\AttributeTok{ncol =} \DecValTok{2}\NormalTok{))}
\end{Highlighting}
\end{Shaded}

\includegraphics{rapport_files/figure-latex/unnamed-chunk-9-1.pdf}

\begin{Shaded}
\begin{Highlighting}[]
\FunctionTok{ggplot}\NormalTok{(big5\_data\_by\_team, }\FunctionTok{aes}\NormalTok{(}\AttributeTok{x =}\NormalTok{ Equipe, }\AttributeTok{y =}\NormalTok{ Touches, }\AttributeTok{fill =}\NormalTok{ Equipe)) }\SpecialCharTok{+} \FunctionTok{geom\_bar}\NormalTok{(}\AttributeTok{stat =} \StringTok{"identity"}\NormalTok{) }\SpecialCharTok{+} \FunctionTok{theme\_minimal}\NormalTok{() }\SpecialCharTok{+} \FunctionTok{labs}\NormalTok{(}\AttributeTok{title =} \StringTok{"Nombre de touches par équipe"}\NormalTok{, }\AttributeTok{x =} \StringTok{"Equipe"}\NormalTok{, }\AttributeTok{y =} \StringTok{"Nombre de touches"}\NormalTok{)}\SpecialCharTok{+}\FunctionTok{coord\_cartesian}\NormalTok{(}\AttributeTok{ylim =} \FunctionTok{c}\NormalTok{(}\DecValTok{40000}\NormalTok{,}\DecValTok{50000}\NormalTok{))}\SpecialCharTok{+}  \FunctionTok{theme}\NormalTok{(}\AttributeTok{axis.text.x =} \FunctionTok{element\_text}\NormalTok{(}\AttributeTok{angle =} \DecValTok{90}\NormalTok{, }\AttributeTok{vjust =} \FloatTok{0.5}\NormalTok{, }\AttributeTok{hjust =} \DecValTok{1}\NormalTok{))}\SpecialCharTok{+}
  \FunctionTok{guides}\NormalTok{(}\AttributeTok{fill =} \FunctionTok{guide\_legend}\NormalTok{(}\AttributeTok{ncol =} \DecValTok{2}\NormalTok{))}
\end{Highlighting}
\end{Shaded}

\includegraphics{rapport_files/figure-latex/unnamed-chunk-9-2.pdf}

\begin{Shaded}
\begin{Highlighting}[]
\FunctionTok{ggplot}\NormalTok{(big5\_data\_by\_team, }\FunctionTok{aes}\NormalTok{(}\AttributeTok{x =}\NormalTok{ Equipe, }\AttributeTok{y =}\NormalTok{ Reussite\_Dribble, }\AttributeTok{fill =}\NormalTok{ Equipe)) }\SpecialCharTok{+} \FunctionTok{geom\_bar}\NormalTok{(}\AttributeTok{stat =} \StringTok{"identity"}\NormalTok{) }\SpecialCharTok{+} \FunctionTok{theme\_minimal}\NormalTok{() }\SpecialCharTok{+} \FunctionTok{labs}\NormalTok{(}\AttributeTok{title =} \StringTok{"Pourcentage de dribbles réussis par équipe"}\NormalTok{, }\AttributeTok{x =} \StringTok{"Equipe"}\NormalTok{, }\AttributeTok{y =} \StringTok{"Pourcentage de dribbles réussis"}\NormalTok{)}\SpecialCharTok{+}  \FunctionTok{theme}\NormalTok{(}\AttributeTok{axis.text.x =} \FunctionTok{element\_text}\NormalTok{(}\AttributeTok{angle =} \DecValTok{90}\NormalTok{, }\AttributeTok{vjust =} \FloatTok{0.5}\NormalTok{, }\AttributeTok{hjust =} \DecValTok{1}\NormalTok{))}\SpecialCharTok{+}
  \FunctionTok{guides}\NormalTok{(}\AttributeTok{fill =} \FunctionTok{guide\_legend}\NormalTok{(}\AttributeTok{ncol =} \DecValTok{2}\NormalTok{))}
\end{Highlighting}
\end{Shaded}

\includegraphics{rapport_files/figure-latex/unnamed-chunk-9-3.pdf}

On constate qu'il n'y a finalement que très peu de différence entre les
équipes pour ces paramètres. On peut donc en conclure que ces paramètres
ne sont pas déterminants pour différencier les équipes entre elles. Cela
peut s'expliquer par le fait que les équipes de haut niveau ont des
joueurs de qualité, et que ces paramètres sont donc assez homogènes
entre les équipes de haut niveau.

On constate finalement que deux paramètres semblent être lié à la
position des équipes dans la ligue : le nombre de but, et le nombre de
touches. Pour les autres, il est possible que le fait que ce sont des
pourcentages de réussites ne permettent pas d'identifier une
corrélation, on peut éventuellement s'intéresser au nombre de dribbles
et de pressing reussis, et non pas aux pourcentages de réussite.

\includegraphics{rapport_files/figure-latex/unnamed-chunk-10-1.pdf}
\includegraphics{rapport_files/figure-latex/unnamed-chunk-10-2.pdf}

À nouveau, on ne constate aucun lien particulier entre ces paramètres et
la position des équipes dans la ligue. On peut donc en conclure que le
nombre de buts et le nombre de touches sont les paramètres les plus
importants pour différencier les équipes entre elles.

On peut maintenant se demander si pour d'autres championnat, comme la
Ligue 1, le nombre de buts et le nombres de touches sont également les
paramètres les plus importants. On effectue donc la même analyse, mais
avec toutes les équipes de la ligue 1 On ne va regarder que le nombre de
buts et le nombre de touches.

\includegraphics{rapport_files/figure-latex/unnamed-chunk-11-1.pdf}
\includegraphics{rapport_files/figure-latex/unnamed-chunk-11-2.pdf}

On constate que les résultats ne sont pas vraiment similaires à ce que
l'on pouvait retrouver pour la Premiere League. En effet, on constate
que le nombre de but ne semble pas réellement dépendre du classement des
équipes, et on remarque également que le nombre de touche est
relativement similaires à l'exception de quelques équipes. On en conclut
donc que les meilleures équipes de championnat se démarque dans des
caractéristiques différentes en fonction du champinnat concerné.

\paragraph{Conclusion sur la première
question}\label{conclusion-sur-la-premiuxe8re-question}

Pour conclure sur cette première question, nous pouvons dire qu'en ce
qui concerne la première ligue, les deux paramètres qui semblent être
déterminant pour différencier une équipe en haut du tableau et une
équipe en bas du tableau sont le nombre de buts marqués et le nombre de
touches.

Cependant, nous avons aussi pu remarquer que beaucoup de
caractéristiques qui semblaient être pertinentes pour différencier les
meilleures équipes des moins bonnes ne peuvent en réalité pas être
réellement exploitées.

Cela nous amène donc à penser qu'au delà du nombre de buts et de
touches, c'est surtout le cumul d'une multitude d'actions faites par les
joueurs durant un match, des actions qui sont difficilement
représentables par des chiffres, et qui sont pourtant décisives pour
différencier une équipe d'une autre. \#\#\# \textbf{Analyse de la
relation entre les salaires, les notes et les postes des joueurs de
Manchester City}

(traitée par Amin)

Dans cette partie, nous explorons la relation entre les salaires
annuels, les notes des joueurs et leurs postes au sein de l'équipe de
Manchester City pour la saison 2023-2024. On peut supposer que les
salaires seront d'autant plus élevées que les performances du joueurs
sont bonnes (notes élevées), nous allons vérifier cela. Quant au poste
du joueur, certains favorisent-ils la rémunération ?

\textbf{Chargement des données} :

\begin{Shaded}
\begin{Highlighting}[]
\CommentTok{\# Chargement du package worldfootballR}
\FunctionTok{library}\NormalTok{(worldfootballR)}
\end{Highlighting}
\end{Shaded}

\begin{verbatim}
## Warning: le package 'worldfootballR' a été compilé avec la version R 4.2.3
\end{verbatim}

\textbf{Chargement des bibliothèques nécessaires} :

\begin{itemize}
\tightlist
\item
  \texttt{dplyr\ et\ tibble} : pour la manipulation des données
\item
  \texttt{ggplot2} : pour la création de graphiques
\item
  \texttt{scales} : pour la mise en forme des échelles sur les
  graphiques
\end{itemize}

\begin{Shaded}
\begin{Highlighting}[]
\CommentTok{\# Charger les bibliothèques nécessaires}
\FunctionTok{library}\NormalTok{(dplyr)}
\FunctionTok{library}\NormalTok{(ggplot2)}
\FunctionTok{library}\NormalTok{(scales)}
\end{Highlighting}
\end{Shaded}

\begin{verbatim}
## Warning: le package 'scales' a été compilé avec la version R 4.2.3
\end{verbatim}

\begin{verbatim}
## 
## Attachement du package : 'scales'
\end{verbatim}

\begin{verbatim}
## L'objet suivant est masqué depuis 'package:readr':
## 
##     col_factor
\end{verbatim}

\begin{Shaded}
\begin{Highlighting}[]
\FunctionTok{library}\NormalTok{(tibble)}
\end{Highlighting}
\end{Shaded}

\begin{verbatim}
## Warning: le package 'tibble' a été compilé avec la version R 4.2.3
\end{verbatim}

Nous avons chargé les données sur les salaires des joueurs de Manchester
City à partir du site FBref. Les données ont été stockées dans un
dataframe nommé \texttt{man\_city\_wages}.

\begin{Shaded}
\begin{Highlighting}[]
\CommentTok{\# Importation des données des joueurs à partir de l\textquotesingle{}URL spécifiée}
\NormalTok{man\_city\_url }\OtherTok{\textless{}{-}} \StringTok{"https://fbref.com/en/squads/b8fd03ef/Manchester{-}City{-}Stats"}
\NormalTok{man\_city\_wages }\OtherTok{\textless{}{-}} \FunctionTok{fb\_squad\_wages}\NormalTok{(}\AttributeTok{team\_urls =}\NormalTok{ man\_city\_url)}
\end{Highlighting}
\end{Shaded}

\textbf{Traitement des données} :

Pour mieux analyser les données, nous avons recodé les postes des
joueurs en de nouveaux groupes en fonction de leur position sur le
terrain. Voici la logique derrière ce recodage : - Les joueurs occupant
les postes ``FW'' (attaquant), ``FW,MF'' (attaquant ou milieu) et
``MF,FW'' (milieu ou attaquant) ont été regroupés sous la catégorie
``Attaquant''. - Ceux occupant les postes ``MF'' (milieu) et ``DF,MF''
(défenseur ou milieu) ont été regroupés sous la catégorie ``Milieu''. -
Les joueurs occupant les postes ``DF'' (défenseur), ``DF,FW'' (défenseur
ou attaquant), ``RB'' (arrière droit) et ``AM'' (ailier) ont été
regroupés sous la catégorie ``Défenseur''. - Les gardiens de but,
identifiés par le poste ``GK'', ont été regroupés dans la catégorie
``Gardien de but''. - Les joueurs occupant d'autres postes ont été
regroupés dans la catégorie ``Autre''.

Cette séquence de code applique la fonction case\_when() pour chaque
ligne du dataframe man\_city\_wages. Selon les valeurs de la colonne Pos
(qui représente les postes des joueurs), cette fonction affecte un
nouveau groupe (Groupe\_poste) à chaque joueur en fonction de son poste
d'origine. Les postes sont reclassés en fonction des critères définis,
et les joueurs sont ainsi regroupés de manière cohérente pour une
analyse plus approfondie.

\begin{Shaded}
\begin{Highlighting}[]
\CommentTok{\# Recoder les postes en nouveaux groupes}
\NormalTok{man\_city\_wages }\OtherTok{\textless{}{-}}\NormalTok{ man\_city\_wages }\SpecialCharTok{\%\textgreater{}\%}
  \FunctionTok{mutate}\NormalTok{(}\AttributeTok{Groupe\_poste =} \FunctionTok{case\_when}\NormalTok{(}
\NormalTok{    Pos }\SpecialCharTok{\%in\%} \FunctionTok{c}\NormalTok{(}\StringTok{"FW"}\NormalTok{, }\StringTok{"FW,MF"}\NormalTok{, }\StringTok{"MF,FW"}\NormalTok{) }\SpecialCharTok{\textasciitilde{}} \StringTok{"Attaquant"}\NormalTok{,}
\NormalTok{    Pos }\SpecialCharTok{\%in\%} \FunctionTok{c}\NormalTok{(}\StringTok{"MF"}\NormalTok{, }\StringTok{"DF,MF"}\NormalTok{) }\SpecialCharTok{\textasciitilde{}} \StringTok{"Milieu"}\NormalTok{,}
\NormalTok{    Pos }\SpecialCharTok{\%in\%} \FunctionTok{c}\NormalTok{(}\StringTok{"DF"}\NormalTok{, }\StringTok{"DF,FW"}\NormalTok{, }\StringTok{"RB"}\NormalTok{, }\StringTok{"AM"}\NormalTok{) }\SpecialCharTok{\textasciitilde{}} \StringTok{"Défenseur"}\NormalTok{,}
\NormalTok{    Pos }\SpecialCharTok{==} \StringTok{"GK"} \SpecialCharTok{\textasciitilde{}} \StringTok{"Gardien de but"}\NormalTok{,}
    \ConstantTok{TRUE} \SpecialCharTok{\textasciitilde{}} \StringTok{"Autre"}
\NormalTok{  ))}
\end{Highlighting}
\end{Shaded}

Création d'un dataframe avec les données de note des joueurs. Nous avons
récupérés ses notes sur le site officiel de la Prmeier League.

Afin de prendre en compte les notes des joueurs, nous avons créé un
nouveau dataframe nommé \texttt{notes\_df}. Ce dataframe contient deux
colonnes : - La colonne \texttt{Joueur}, qui indique le nom de chaque
joueur. - La colonne \texttt{Note}, qui représente la note attribuée à
chaque joueur.

Voici un échantillon des joueurs et leurs notes correspondantes : -
Kevin De Bruyne : 7.60 - Erling Haaland : 7.35 - Bernardo Silva : 7.10 -
\ldots{}

On remarque que des notes de joueurs sont manquants. Ces notes sont des
données importantes pour comprendre et analyser la performance des
joueurs.

\begin{Shaded}
\begin{Highlighting}[]
\CommentTok{\# Créer un dataframe avec les données de note}
\NormalTok{notes\_df }\OtherTok{\textless{}{-}} \FunctionTok{tribble}\NormalTok{(}
  \SpecialCharTok{\textasciitilde{}}\NormalTok{Joueur,               }\SpecialCharTok{\textasciitilde{}}\NormalTok{Note,}
  \StringTok{"Kevin De Bruyne"}\NormalTok{,     }\FloatTok{7.60}\NormalTok{,}
  \StringTok{"Erling Haaland"}\NormalTok{,      }\FloatTok{7.35}\NormalTok{,}
  \StringTok{"Bernardo Silva"}\NormalTok{,      }\FloatTok{7.10}\NormalTok{,}
  \StringTok{"Jack Grealish"}\NormalTok{,       }\FloatTok{6.75}\NormalTok{,}
  \StringTok{"John Stones"}\NormalTok{,         }\FloatTok{6.58}\NormalTok{,}
  \StringTok{"Phil Foden"}\NormalTok{,          }\FloatTok{7.26}\NormalTok{,}
  \StringTok{"Rodri"}\NormalTok{,               }\FloatTok{7.59}\NormalTok{,}
  \StringTok{"Joško Gvardiol"}\NormalTok{,      }\FloatTok{6.93}\NormalTok{,}
  \StringTok{"Rúben Dias"}\NormalTok{,          }\FloatTok{6.80}\NormalTok{,}
  \StringTok{"Manuel Akanji"}\NormalTok{,       }\FloatTok{6.81}\NormalTok{,}
  \StringTok{"Kyle Walker"}\NormalTok{,         }\FloatTok{6.84}\NormalTok{,}
  \StringTok{"Nathan Aké"}\NormalTok{,          }\FloatTok{6.78}\NormalTok{,}
  \StringTok{"Kalvin Phillips"}\NormalTok{,     }\FloatTok{6.27}\NormalTok{,}
  \StringTok{"Mateo Kovačić"}\NormalTok{,       }\FloatTok{6.62}\NormalTok{,}
  \StringTok{"Matheus Nunes"}\NormalTok{,       }\FloatTok{6.37}\NormalTok{,}
  \StringTok{"Ederson"}\NormalTok{,             }\FloatTok{6.60}\NormalTok{,}
  \StringTok{"Julián Álvarez"}\NormalTok{,      }\FloatTok{7.15}\NormalTok{,}
  \StringTok{"Stefan Ortega"}\NormalTok{,       }\FloatTok{6.67}\NormalTok{,}
  \StringTok{"Jeremy Doku"}\NormalTok{,         }\FloatTok{7.26}\NormalTok{,}
  \StringTok{"Sergio Gómez"}\NormalTok{,        }\FloatTok{6.24}\NormalTok{,}
  \StringTok{"Zack Steffen"}\NormalTok{,        }\FloatTok{6.60}\NormalTok{,}
  \StringTok{"Scott Carson"}\NormalTok{,        }\ConstantTok{NA}\NormalTok{,}
  \StringTok{"Rico Lewis"}\NormalTok{,          }\FloatTok{6.56}\NormalTok{,}
  \StringTok{"Oscar Bobb"}\NormalTok{,          }\ConstantTok{NA}
\NormalTok{)}
\end{Highlighting}
\end{Shaded}

Ajout de la colonne ``Note'' à man\_city\_wages.

Pour enrichir le dataframe \texttt{man\_city\_wages} avec les notes des
joueurs, nous avons ajouté une nouvelle colonne appelée ``Note''. Nous
avons utilisé la fonction \texttt{match} pour faire correspondre les
noms des joueurs dans le dataframe \texttt{man\_city\_wages} avec ceux
du dataframe \texttt{notes\_df}, afin d'obtenir les notes
correspondantes.

Voici les étapes que nous avons suivies : 1. Nous avons utilisé la
fonction \texttt{match} pour trouver les indices des noms de joueurs
dans le dataframe \texttt{notes\_df}. 2. Nous avons récupéré les notes
correspondantes à ces indices dans la colonne ``Note'' du dataframe
\texttt{notes\_df}. 3. Nous avons ajouté ces notes comme nouvelle
colonne ``Note'' dans le dataframe \texttt{man\_city\_wages}. 4. Enfin,
nous avons supprimé les colonnes non pertinentes telles que ``Url'',
``Notes'', ``AnnualWageGBP'', ``AnnualWageUSD'', ``WeeklyWageUSD'',
``WeeklyWageGBP'', et ``Comp'' du dataframe \texttt{man\_city\_wages} à
l'aide de la fonction \texttt{select}.

\begin{Shaded}
\begin{Highlighting}[]
\CommentTok{\# Ajouter la colonne "Note" à man\_city\_wages}
\NormalTok{man\_city\_wages}\SpecialCharTok{$}\NormalTok{Note }\OtherTok{\textless{}{-}}\NormalTok{ notes\_df}\SpecialCharTok{$}\NormalTok{Note[}\FunctionTok{match}\NormalTok{(man\_city\_wages}\SpecialCharTok{$}\NormalTok{Player, notes\_df}\SpecialCharTok{$}\NormalTok{Joueur)]}
\NormalTok{man\_city\_wages }\OtherTok{\textless{}{-}} \FunctionTok{select}\NormalTok{(man\_city\_wages, }\SpecialCharTok{{-}}\NormalTok{Url, }\SpecialCharTok{{-}}\NormalTok{Notes, }\SpecialCharTok{{-}}\NormalTok{AnnualWageGBP, }\SpecialCharTok{{-}}\NormalTok{AnnualWageUSD, }\SpecialCharTok{{-}}\NormalTok{WeeklyWageUSD, }\SpecialCharTok{{-}}\NormalTok{WeeklyWageGBP, }\SpecialCharTok{{-}}\NormalTok{Comp)}
\end{Highlighting}
\end{Shaded}

Cela nous permet d'avoir un dataframe complet comprenant les salaires
annuels, les postes et les notes des joueurs de Manchester City.

\subsubsection{Graphique}\label{graphique}

Nous avons créé un graphique pour explorer la relation entre les
salaires annuels, les notes et les postes des joueurs de Manchester
City.

\textbf{Description du graphique} : - L'axe des abscisses représente les
notes des joueurs. - L'axe des ordonnées représente les salaires annuels
des joueurs en euros. - Les points sur le graphique représentent les
joueurs, où chaque point est placé en fonction de sa note et de son
salaire annuel. - Les formes des points sont déterminées par les groupes
de postes des joueurs. - Les couleurs des points représentent également
les groupes de postes des joueurs. - Nous avons utilisé des formes
différentes pour représenter les différents postes des joueurs. - Le
titre du graphique est ``Relation entre les salaires annuels, les notes
et les postes des joueurs de Manchester City''. - Les étiquettes des
axes sont ``Note'' pour l'axe des abscisses et ``Salaire annuel (en
euros)'' pour l'axe des ordonnées.

\begin{Shaded}
\begin{Highlighting}[]
\CommentTok{\# Visualisation : Comparaison des salaires annuels des joueurs de Manchester City en fonction des notes et des postes}
\FunctionTok{ggplot}\NormalTok{(man\_city\_wages, }\FunctionTok{aes}\NormalTok{(}\AttributeTok{x =}\NormalTok{ Note, }\AttributeTok{y =}\NormalTok{ AnnualWageEUR, }\AttributeTok{shape =}\NormalTok{ Groupe\_poste, }\AttributeTok{color =}\NormalTok{ Groupe\_poste)) }\SpecialCharTok{+}
  \FunctionTok{geom\_point}\NormalTok{(}\AttributeTok{size =} \DecValTok{4}\NormalTok{, }\AttributeTok{alpha =} \FloatTok{0.7}\NormalTok{) }\SpecialCharTok{+}
  \FunctionTok{labs}\NormalTok{(}\AttributeTok{title =} \StringTok{"Relation entre les salaires annuels, les notes et les postes des joueurs de Manchester City"}\NormalTok{,}
       \AttributeTok{x =} \StringTok{"Note"}\NormalTok{,}
       \AttributeTok{y =} \StringTok{"Salaire annuel (en euros)"}\NormalTok{,}
       \AttributeTok{shape =} \StringTok{"Poste"}\NormalTok{,}
       \AttributeTok{color =} \StringTok{"Poste"}\NormalTok{) }\SpecialCharTok{+}
  \FunctionTok{scale\_x\_continuous}\NormalTok{(}\AttributeTok{breaks =} \FunctionTok{seq}\NormalTok{(}\DecValTok{5}\NormalTok{, }\DecValTok{10}\NormalTok{, }\AttributeTok{by =} \FloatTok{0.5}\NormalTok{)) }\SpecialCharTok{+}
  \FunctionTok{scale\_y\_continuous}\NormalTok{(}\AttributeTok{labels =}\NormalTok{ scales}\SpecialCharTok{::}\NormalTok{comma) }\SpecialCharTok{+}
  \FunctionTok{scale\_shape\_manual}\NormalTok{(}\AttributeTok{values =} \FunctionTok{c}\NormalTok{(}\DecValTok{17}\NormalTok{, }\DecValTok{19}\NormalTok{, }\DecValTok{15}\NormalTok{, }\DecValTok{16}\NormalTok{, }\DecValTok{18}\NormalTok{, }\DecValTok{3}\NormalTok{, }\DecValTok{4}\NormalTok{)) }\SpecialCharTok{+}
  \FunctionTok{scale\_color\_brewer}\NormalTok{(}\AttributeTok{palette =} \StringTok{"Set1"}\NormalTok{) }\SpecialCharTok{+}
  \FunctionTok{theme\_minimal}\NormalTok{() }\SpecialCharTok{+}
  \FunctionTok{theme}\NormalTok{(}\AttributeTok{legend.title =} \FunctionTok{element\_text}\NormalTok{(}\AttributeTok{face =} \StringTok{"bold"}\NormalTok{),}
        \AttributeTok{legend.position =} \StringTok{"top"}\NormalTok{,}
        \AttributeTok{legend.direction =} \StringTok{"horizontal"}\NormalTok{,}
        \AttributeTok{legend.key.size =} \FunctionTok{unit}\NormalTok{(}\FloatTok{1.5}\NormalTok{, }\StringTok{"lines"}\NormalTok{))}
\end{Highlighting}
\end{Shaded}

\begin{verbatim}
## Warning: Removed 2 rows containing missing values or values outside the scale range
## (`geom_point()`).
\end{verbatim}

\includegraphics{rapport_files/figure-latex/unnamed-chunk-18-1.pdf}

\textbf{Analyse finale} :

L'analyse révèle une corrélation claire entre les notes des joueurs et
leurs salaires à Manchester City. Les joueurs mieux notés ont tendance à
percevoir des salaires plus élevés, mettant en lumière l'impact des
performances individuelles sur la rémunération. De plus, les attaquants
et les milieux de terrain sont les mieux rémunérés, soulignant
l'importance de ces postes dans le jeu. Les gardiens de but présentent
des notes et des salaires similaires, reflétant une certaine équité dans
la rémunération de ce poste. Enfin, cette analyse met en évidence
l'accent mis sur l'offensive dans le football professionnel, où les
attaquants sont souvent les mieux notés et rémunérés.

\subsubsection{\texorpdfstring{\textbf{Quelles sont les blessures les
plus fréquentes pour les joueurs de Premier League
?}}{Quelles sont les blessures les plus fréquentes pour les joueurs de Premier League ?}}\label{quelles-sont-les-blessures-les-plus-fruxe9quentes-pour-les-joueurs-de-premier-league}

(traité par Ahmed)

Dans cette analyse, nous chercherons à identifier les types de blessures
les plus courantes et les parties du corps les plus affectées chez les
joueurs professionnels, en mettant particulièrement l'accent sur les
joueurs de Premier League. Nous pourrions nous attendre à ce que les
blessures aux chevilles, aux genoux et aux adducteurs soient parmi les
plus fréquentes.

\textbf{Traitement des données} : Dans le cadre de notre analyse des
blessures des joueurs de Premier League, nous avons collecté des données
à partir de différentes sources. Tout d'abord, nous avons créé une liste
de tous les clubs de Premier League, ``clubs\_list'', contenant 26
clubs.

Ensuite, nous avons importé les données relatives aux joueurs à partir
du fichier CSV ``players.csv'' en utilisant la fonction
readr::read\_csv().

\begin{Shaded}
\begin{Highlighting}[]
\NormalTok{players }\OtherTok{\textless{}{-}}\NormalTok{ readr}\SpecialCharTok{::}\FunctionTok{read\_csv}\NormalTok{(}\StringTok{"dataset/playerslinkALL/players.csv"}\NormalTok{)}
\FunctionTok{colnames}\NormalTok{(players)}
\end{Highlighting}
\end{Shaded}

Pour obtenir les statistiques de blessures de chaque joueur, nous avons
utilisé une fonction personnalisée, tm\_player\_injury\_history(), qui
prend l'URL du profil du joueur sur le site web de Transfermarkt comme
argument. Nous avons itéré sur chaque ligne de notre jeu de données
``players'' en utilisant une boucle for, et pour chaque joueur, nous
avons récupéré ses statistiques de blessures en utilisant cette
fonction.

\begin{Shaded}
\begin{Highlighting}[]
\NormalTok{players\_stats\_list }\OtherTok{\textless{}{-}} \FunctionTok{list}\NormalTok{()}

\ControlFlowTok{for}\NormalTok{ (i }\ControlFlowTok{in} \DecValTok{1}\SpecialCharTok{:}\FunctionTok{nrow}\NormalTok{(players)) \{}
\NormalTok{  player\_url }\OtherTok{\textless{}{-}}\NormalTok{ players}\SpecialCharTok{$}\NormalTok{UrlTmarkt[i]}
\NormalTok{  stat }\OtherTok{\textless{}{-}} \FunctionTok{tm\_player\_injury\_history}\NormalTok{(player\_url)}
  \ControlFlowTok{if}\NormalTok{ (}\FunctionTok{any}\NormalTok{(}\SpecialCharTok{!}\FunctionTok{is.na}\NormalTok{(stat}\SpecialCharTok{$}\NormalTok{club))) \{}
    \FunctionTok{print}\NormalTok{(}\StringTok{"club"}\NormalTok{)}
\NormalTok{    found }\OtherTok{\textless{}{-}} \ConstantTok{FALSE}
\NormalTok{    clubs }\OtherTok{\textless{}{-}} \FunctionTok{strsplit}\NormalTok{(stat}\SpecialCharTok{$}\NormalTok{club, }\StringTok{", "}\NormalTok{)[[}\DecValTok{1}\NormalTok{]]}
    \ControlFlowTok{for}\NormalTok{ (club }\ControlFlowTok{in}\NormalTok{ clubs) \{}
\NormalTok{      found }\OtherTok{\textless{}{-}} \FunctionTok{grepl}\NormalTok{(club, clubs\_list, }\AttributeTok{ignore.case =} \ConstantTok{TRUE}\NormalTok{)}
      \ControlFlowTok{if}\NormalTok{ (}\FunctionTok{any}\NormalTok{(found)) \{}
        \FunctionTok{print}\NormalTok{(}\StringTok{"en PL"}\NormalTok{)}
\NormalTok{        players\_stats\_list[[i]] }\OtherTok{\textless{}{-}}\NormalTok{ stat}
        \ControlFlowTok{break}
\NormalTok{      \} }\ControlFlowTok{else}\NormalTok{ \{}
        \FunctionTok{print}\NormalTok{(}\StringTok{"pas en PL "}\NormalTok{)}
\NormalTok{      \}}
\NormalTok{    \}}
\NormalTok{  \}}
\NormalTok{\}}
\end{Highlighting}
\end{Shaded}

Comme un joueur peut avoir joué pour plusieurs clubs au cours de sa
carrière, nous avons vérifié s'il a joué pour un club de Premier League
en comparant le nom du club dans les statistiques de blessures avec
notre liste de clubs de Premier League. Nous avons utilisé la fonction
strsplit() pour diviser la chaîne de caractères contenant les noms des
clubs en un vecteur, puis nous avons itéré sur chaque club en utilisant
une boucle for. Nous avons utilisé la fonction grepl() pour vérifier si
le nom du club est dans notre liste de clubs de Premier League, en
ignorant la casse. Si le joueur a joué pour un club de Premier League,
nous avons ajouté ses statistiques de blessures à notre liste
``players\_stats\_list''.

\begin{Shaded}
\begin{Highlighting}[]
\NormalTok{players\_stats\_df }\OtherTok{\textless{}{-}} \FunctionTok{do.call}\NormalTok{(rbind, players\_stats\_list)}
\end{Highlighting}
\end{Shaded}

\subsubsection{Graphique}\label{graphique-1}

Cette visualisation nous montre les blessures les plus récurrentes avec
un double encodage pour représenter la fréquence, la taille et la
couleur. Ce graphique a été réalisé à partir de données recensant 1000
blessures de joueurs ayant évolué en Premier League.

\subsubsection{Analyse des graphiques}\label{analyse-des-graphiques}

Les résultats confirment nos attentes, avec comme top 3 des blessures
les plus récurrentes : en premier, les adducteurs, suivis des genoux,
puis des chevilles. Cette observation est logique étant donné que le
football de haut niveau exerce une pression particulière sur certains
muscles et articulations. Nous remarquons également l'impact de la crise
du COVID-19, qui a réussi à figurer parmi les blessures les plus
récurrentes, bien que son impact ait été de courte durée dans les
rapports de blessures (avant la suspension du championnat).

\subsubsection{\texorpdfstring{\textbf{Le nombre de blessures est-il lié
au nombre de match joué ou le championnat joue une plus grosse partie
(stéréotype de championnat + physique que d'autres
)}}{Le nombre de blessures est-il lié au nombre de match joué ou le championnat joue une plus grosse partie (stéréotype de championnat + physique que d'autres )}}\label{le-nombre-de-blessures-est-il-liuxe9-au-nombre-de-match-jouuxe9-ou-le-championnat-joue-une-plus-grosse-partie-stuxe9ruxe9otype-de-championnat-physique-que-dautres}

(traité par Ahmed) \textbf{optionnel dans la deadline du 5 mai}

Dans cette analyse, nous nous intéressons à la relation entre le nombre
de jours entre chaque match et le nombre de blessures dans les ligues
majeures de football. Notre question de recherche est de savoir si
certains championnats sont plus sujets aux blessures que d'autres en
raison de leur nature plus physique supposée . Dans l'imaginaire
collectif des fans de football, la Premier League a une réputation de
championnat plus physique que les autres. On pourrait donc s'attendre à
plus de blessures de la part des joueurs de cette ligue.

\subsubsection{Graphiques}\label{graphiques}

Sur ce graphique, l'axe des abscisses représente le nombre moyen de
jours entre chaque match, tandis que l'axe des ordonnées représente le
nombre de blessures. Nous observons une tendance générale où les
championnats avec moins de jours de repos entre les matchs ont tendance
à avoir un nombre de blessures plus élevé.

Sur ce graphique, l'axe des abscisses représente le nombre de matchs par
équipe en championnat, tandis que l'axe des ordonnées représente le
nombre de blessures. Nous observons une relation linéaire entre le
nombre de matchs et le nombre de blessures, sauf pour la Ligue 1 qui
semble être une exception à cette tendance.

\subsubsection{Analyse des graphiques}\label{analyse-des-graphiques-1}

Les graphiques montrent clairement des tendances. Nous constatons
presque une relation linéaire entre le nombre de matchs et les
blessures, à l'exception de la Ligue 1. Cela confirme l'idée que
certains championnats peuvent être plus sujets aux blessures en raison
de leur intensité de matchs.

En ce qui concerne la relation entre le délai entre chaque match et les
blessures, nous observons que les délais sont souvent très proches, mais
nous remarquons deux extrêmes. La Premier League a le plus grand nombre
de blessures avec le moins de temps de repos, tandis que la Ligue 1 a le
nombre de blessures le plus bas avec le temps de repos le plus long.
Cela suggère que plus une équipe a de temps de repos, moins elle est
exposée aux blessures.

Cependant, la Bundesliga se distingue en ayant un bon temps de repos
mais un nombre élevé de blessures. Cela soulève la question de savoir
quels autres facteurs influencent le nombre de blessures, en dehors du
nombre de matchs. Il est important de prendre en compte que le nombre de
matchs peut varier selon les équipes au sein d'un même championnat, en
fonction de leur participation dans les compétitions de coupe.

En conclusion, ces observations soulignent l'importance du repos entre
les matchs dans la prévention des blessures, mais elles soulèvent
également d'autres questions sur les facteurs influençant le nombre de
blessures dans le football professionnel car le repos n'étant pas
l'indice absolu . On pourrait donc se demander quels sont ces autres
indices ?

\subsubsection{\texorpdfstring{\textbf{Déterminer le profil de l'équipe
parfaite (possession par
exemple)}}{Déterminer le profil de l'équipe parfaite (possession par exemple)}}\label{duxe9terminer-le-profil-de-luxe9quipe-parfaite-possession-par-exemple}

(traité par nassim)

\section{Introduction}\label{introduction-1}

La possession de balle est souvent citée comme un indicateur clé de la
performance d'une équipe de football. Les équipes qui dominent la
possession sont souvent perçues comme ayant un meilleur contrôle du jeu,
une défense plus solide et une attaque plus efficace. Ce rapport vise à
analyser en profondeur l'impact de la possession de balle sur les
performances des équipes à travers cinq graphiques distincts, et à
déterminer le profil de l'équipe parfaite en termes de possession de
balle.

\section{Question de Recherche}\label{question-de-recherche}

\textbf{Question :} Déterminer le profil de l'équipe parfaite en termes
de possession de balle.

\section{Hypothèses}\label{hypothuxe8ses}

\begin{enumerate}
\def\labelenumi{\arabic{enumi}.}
\tightlist
\item
  \textbf{Hypothèse 1 :} Les équipes avec une plus grande possession de
  balle ont tendance à marquer plus de buts par match.
\item
  \textbf{Hypothèse 2 :} Les équipes qui dominent en possession de balle
  sont plus efficaces dans la création d'occasions de but.
\item
  \textbf{Hypothèse 3 :} La possession de balle est liée à la capacité
  d'une équipe à contrôler le jeu.
\end{enumerate}

\section{Analyse des Graphiques}\label{analyse-des-graphiques-2}

Nous avons analysé cinq graphiques pour vérifier les hypothèses et
déterminer le profil de l'équipe parfaite en termes de possession de
balle.

\subsection{Graphique 1 : Statistiques des Équipes Championnes des 5
Grandes Ligues
(2023)}\label{graphique-1-statistiques-des-uxe9quipes-championnes-des-5-grandes-ligues-2023}

Ce graphique en barres affiche le pourcentage de possession, les buts
marqués (Buts mis) et les buts encaissés (Buts pris) des équipes
championnes des cinq grandes ligues européennes en 2023. Les ligues sont
codées par couleur, et les équipes sont listées comme suit :

\begin{itemize}
\tightlist
\item
  \textbf{Paris S-G (Ligue 1)}

  \begin{itemize}
  \tightlist
  \item
    Buts marqués : 86
  \item
    Buts encaissés : 39
  \item
    Possession : 60.7\%
  \end{itemize}
\item
  \textbf{Napoli (Serie A)}

  \begin{itemize}
  \tightlist
  \item
    Buts marqués : 75
  \item
    Buts encaissés : 28
  \item
    Possession : 61.8\%
  \end{itemize}
\item
  \textbf{Bayern Munich (Bundesliga)}

  \begin{itemize}
  \tightlist
  \item
    Buts marqués : 90
  \item
    Buts encaissés : 38
  \item
    Possession : 64.2\%
  \end{itemize}
\item
  \textbf{Barcelona (La Liga)}

  \begin{itemize}
  \tightlist
  \item
    Buts marqués : 69
  \item
    Buts encaissés : 17
  \item
    Possession : 64.3\%
  \end{itemize}
\item
  \textbf{Manchester City (Premier League)}

  \begin{itemize}
  \tightlist
  \item
    Buts marqués : 92
  \item
    Buts encaissés : 32
  \item
    Possession : 64.7\%
  \end{itemize}
\end{itemize}

\begin{Shaded}
\begin{Highlighting}[]
\NormalTok{champions }\OtherTok{\textless{}{-}} \FunctionTok{data.frame}\NormalTok{(}
\NormalTok{  é}\AttributeTok{quipe =} \FunctionTok{c}\NormalTok{(}\StringTok{"Paris S{-}G"}\NormalTok{, }\StringTok{"Napoli"}\NormalTok{, }\StringTok{"Bayern Munich"}\NormalTok{, }\StringTok{"Barcelona"}\NormalTok{, }\StringTok{"Manchester City"}\NormalTok{),}
  \AttributeTok{buts\_mis =} \FunctionTok{c}\NormalTok{(}\DecValTok{86}\NormalTok{, }\DecValTok{75}\NormalTok{, }\DecValTok{90}\NormalTok{, }\DecValTok{69}\NormalTok{, }\DecValTok{92}\NormalTok{),}
  \AttributeTok{buts\_pris =} \FunctionTok{c}\NormalTok{(}\DecValTok{39}\NormalTok{, }\DecValTok{28}\NormalTok{, }\DecValTok{38}\NormalTok{, }\DecValTok{17}\NormalTok{, }\DecValTok{32}\NormalTok{),}
  \AttributeTok{possession =} \FunctionTok{c}\NormalTok{(}\FloatTok{60.7}\NormalTok{, }\FloatTok{61.8}\NormalTok{, }\FloatTok{64.2}\NormalTok{, }\FloatTok{64.3}\NormalTok{, }\FloatTok{64.7}\NormalTok{),}
\NormalTok{  compétition }\OtherTok{=} \FunctionTok{c}\NormalTok{(}\StringTok{"Ligue 1"}\NormalTok{, }\StringTok{"Serie A"}\NormalTok{, }\StringTok{"Bundesliga"}\NormalTok{, }\StringTok{"La Liga"}\NormalTok{, }\StringTok{"Premier League"}\NormalTok{)}
\NormalTok{)}

\FunctionTok{ggplot}\NormalTok{(champions, }\FunctionTok{aes}\NormalTok{(}\AttributeTok{x =}\NormalTok{ équipe, }\AttributeTok{y =}\NormalTok{ possession, }\AttributeTok{fill =}\NormalTok{ compétition)) }\SpecialCharTok{+}
  \FunctionTok{geom\_bar}\NormalTok{(}\AttributeTok{stat =} \StringTok{"identity"}\NormalTok{) }\SpecialCharTok{+}
  \FunctionTok{geom\_text}\NormalTok{(}\FunctionTok{aes}\NormalTok{(}\AttributeTok{label =} \FunctionTok{paste}\NormalTok{(}\StringTok{"Buts mis: "}\NormalTok{, buts\_mis, }\StringTok{"}\SpecialCharTok{\textbackslash{}n}\StringTok{Buts pris: "}\NormalTok{, buts\_pris, }\StringTok{"}\SpecialCharTok{\textbackslash{}n}\StringTok{Possession: "}\NormalTok{, possession, }\StringTok{"\%"}\NormalTok{)), }
            \AttributeTok{hjust =} \FloatTok{0.5}\NormalTok{, }\AttributeTok{color =} \StringTok{"black"}\NormalTok{, }\AttributeTok{position =} \FunctionTok{position\_stack}\NormalTok{(}\AttributeTok{vjust =} \FloatTok{0.5}\NormalTok{)) }\SpecialCharTok{+}
  \FunctionTok{labs}\NormalTok{(}\AttributeTok{title =} \StringTok{"Statistiques des équipes championnes des 5 grandes ligues (2023)"}\NormalTok{,}
       \AttributeTok{subtitle =} \StringTok{"Inclut la possession, les buts marqués et les buts encaissés"}\NormalTok{,}
       \AttributeTok{x =} \StringTok{"Équipe"}\NormalTok{,}
       \AttributeTok{y =} \StringTok{"Possession (\%)"}\NormalTok{)}
\end{Highlighting}
\end{Shaded}

\includegraphics{rapport_files/figure-latex/unnamed-chunk-22-1.pdf}

\subsection{Graphique 2 : Contraste Entre les 10 Meilleures et les 10
Pires Équipes en Termes de Buts Marqués par
Possession}\label{graphique-2-contraste-entre-les-10-meilleures-et-les-10-pires-uxe9quipes-en-termes-de-buts-marquuxe9s-par-possession}

Ce graphique en barres montre le ratio de buts marqués par possession
pour les 10 meilleures et 10 pires équipes. Les 10 meilleures équipes
sont :

\begin{itemize}
\tightlist
\item
  \textbf{Monaco (Ligue 1)}

  \begin{itemize}
  \tightlist
  \item
    Ratio : 1.43
  \end{itemize}
\item
  \textbf{Manchester City (Premier League)}

  \begin{itemize}
  \tightlist
  \item
    Ratio : 1.42
  \end{itemize}
\item
  \textbf{Paris S-G (Ligue 1)}

  \begin{itemize}
  \tightlist
  \item
    Ratio : 1.42
  \end{itemize}
\item
  \textbf{Arsenal (Premier League)}

  \begin{itemize}
  \tightlist
  \item
    Ratio : 1.42
  \end{itemize}
\item
  \textbf{Bayern Munich (Bundesliga)}

  \begin{itemize}
  \tightlist
  \item
    Ratio : 1.40
  \end{itemize}
\item
  \textbf{Montpellier (Ligue 1)}

  \begin{itemize}
  \tightlist
  \item
    Ratio : 1.40
  \end{itemize}
\item
  \textbf{Dortmund (Bundesliga)}

  \begin{itemize}
  \tightlist
  \item
    Ratio : 1.39
  \end{itemize}
\item
  \textbf{Tottenham (Premier League)}

  \begin{itemize}
  \tightlist
  \item
    Ratio : 1.36
  \end{itemize}
\item
  \textbf{Atlético Madrid (La Liga)}

  \begin{itemize}
  \tightlist
  \item
    Ratio : 1.34
  \end{itemize}
\item
  \textbf{Atalanta (Serie A)}

  \begin{itemize}
  \tightlist
  \item
    Ratio : 1.28
  \end{itemize}
\end{itemize}

Les 10 pires équipes sont :

\begin{itemize}
\tightlist
\item
  \textbf{Schalke 04 (Bundesliga)}

  \begin{itemize}
  \tightlist
  \item
    Ratio : 0.70
  \end{itemize}
\item
  \textbf{Cádiz (La Liga)}

  \begin{itemize}
  \tightlist
  \item
    Ratio : 0.69
  \end{itemize}
\item
  \textbf{Angers (Ligue 1)}

  \begin{itemize}
  \tightlist
  \item
    Ratio : 0.66
  \end{itemize}
\item
  \textbf{Elche (La Liga)}

  \begin{itemize}
  \tightlist
  \item
    Ratio : 0.66
  \end{itemize}
\item
  \textbf{Valladolid (La Liga)}

  \begin{itemize}
  \tightlist
  \item
    Ratio : 0.66
  \end{itemize}
\item
  \textbf{Chelsea (Premier League)}

  \begin{itemize}
  \tightlist
  \item
    Ratio : 0.63
  \end{itemize}
\item
  \textbf{Spezia (Serie A)}

  \begin{itemize}
  \tightlist
  \item
    Ratio : 0.61
  \end{itemize}
\item
  \textbf{Wolves (Premier League)}

  \begin{itemize}
  \tightlist
  \item
    Ratio : 0.56
  \end{itemize}
\item
  \textbf{Ajaccio (Ligue 1)}

  \begin{itemize}
  \tightlist
  \item
    Ratio : 0.51
  \end{itemize}
\item
  \textbf{Sampdoria (Serie A)}

  \begin{itemize}
  \tightlist
  \item
    Ratio : 0.50
  \end{itemize}
\end{itemize}

\begin{Shaded}
\begin{Highlighting}[]
\NormalTok{top\_bottom\_teams }\OtherTok{\textless{}{-}} \FunctionTok{data.frame}\NormalTok{(}
\NormalTok{  é}\AttributeTok{quipe =} \FunctionTok{c}\NormalTok{(}\StringTok{"Monaco"}\NormalTok{, }\StringTok{"Manchester City"}\NormalTok{, }\StringTok{"Paris S{-}G"}\NormalTok{, }\StringTok{"Arsenal"}\NormalTok{, }\StringTok{"Bayern Munich"}\NormalTok{, }\StringTok{"Montpellier"}\NormalTok{, }\StringTok{"Dortmund"}\NormalTok{, }
             \StringTok{"Tottenham"}\NormalTok{, }\StringTok{"Atlético Madrid"}\NormalTok{, }\StringTok{"Atalanta"}\NormalTok{, }\StringTok{"Schalke 04"}\NormalTok{, }\StringTok{"Cádiz"}\NormalTok{, }\StringTok{"Angers"}\NormalTok{, }\StringTok{"Elche"}\NormalTok{, }\StringTok{"Valladolid"}\NormalTok{, }
             \StringTok{"Chelsea"}\NormalTok{, }\StringTok{"Spezia"}\NormalTok{, }\StringTok{"Wolves"}\NormalTok{, }\StringTok{"Ajaccio"}\NormalTok{, }\StringTok{"Sampdoria"}\NormalTok{),}
  \AttributeTok{ratio =} \FunctionTok{c}\NormalTok{(}\FloatTok{1.43}\NormalTok{, }\FloatTok{1.42}\NormalTok{, }\FloatTok{1.42}\NormalTok{, }\FloatTok{1.42}\NormalTok{, }\FloatTok{1.40}\NormalTok{, }\FloatTok{1.40}\NormalTok{, }\FloatTok{1.39}\NormalTok{, }\FloatTok{1.36}\NormalTok{, }\FloatTok{1.34}\NormalTok{, }\FloatTok{1.28}\NormalTok{, }\FloatTok{0.70}\NormalTok{, }\FloatTok{0.69}\NormalTok{, }\FloatTok{0.66}\NormalTok{, }\FloatTok{0.66}\NormalTok{, }\FloatTok{0.66}\NormalTok{, }\FloatTok{0.63}\NormalTok{, }
            \FloatTok{0.61}\NormalTok{, }\FloatTok{0.56}\NormalTok{, }\FloatTok{0.51}\NormalTok{, }\FloatTok{0.50}\NormalTok{),}
  \AttributeTok{groupe =} \FunctionTok{c}\NormalTok{(}\FunctionTok{rep}\NormalTok{(}\StringTok{"Top 10"}\NormalTok{, }\DecValTok{10}\NormalTok{), }\FunctionTok{rep}\NormalTok{(}\StringTok{"Bottom 10"}\NormalTok{, }\DecValTok{10}\NormalTok{)),}
  \AttributeTok{ligue =} \FunctionTok{c}\NormalTok{(}\StringTok{"Ligue 1"}\NormalTok{, }\StringTok{"Premier League"}\NormalTok{, }\StringTok{"Ligue 1"}\NormalTok{, }\StringTok{"Premier League"}\NormalTok{, }\StringTok{"Bundesliga"}\NormalTok{, }\StringTok{"Ligue 1"}\NormalTok{, }\StringTok{"Bundesliga"}\NormalTok{, }
            \StringTok{"Premier League"}\NormalTok{, }\StringTok{"La Liga"}\NormalTok{, }\StringTok{"Serie A"}\NormalTok{, }\StringTok{"Bundesliga"}\NormalTok{, }\StringTok{"La Liga"}\NormalTok{, }\StringTok{"Ligue 1"}\NormalTok{, }\StringTok{"La Liga"}\NormalTok{, }\StringTok{"La Liga"}\NormalTok{, }
            \StringTok{"Premier League"}\NormalTok{, }\StringTok{"Serie A"}\NormalTok{, }\StringTok{"Premier League"}\NormalTok{, }\StringTok{"Ligue 1"}\NormalTok{, }\StringTok{"Serie A"}\NormalTok{)}
\NormalTok{)}

\FunctionTok{ggplot}\NormalTok{(top\_bottom\_teams, }\FunctionTok{aes}\NormalTok{(}\AttributeTok{x =} \FunctionTok{reorder}\NormalTok{(équipe, }\SpecialCharTok{{-}}\NormalTok{ratio), }\AttributeTok{y =}\NormalTok{ ratio, }\AttributeTok{fill =}\NormalTok{ groupe)) }\SpecialCharTok{+}
  \FunctionTok{geom\_bar}\NormalTok{(}\AttributeTok{stat =} \StringTok{"identity"}\NormalTok{) }\SpecialCharTok{+}
  \FunctionTok{geom\_text}\NormalTok{(}\FunctionTok{aes}\NormalTok{(}\AttributeTok{label =}\NormalTok{ ratio), }\AttributeTok{vjust =} \SpecialCharTok{{-}}\FloatTok{0.5}\NormalTok{, }\AttributeTok{color =} \StringTok{"black"}\NormalTok{) }\SpecialCharTok{+}
  \FunctionTok{labs}\NormalTok{(}\AttributeTok{title =} \StringTok{"Contraste entre les 10 meilleures et les 10 pires équipes en termes de buts marqués par possession"}\NormalTok{,}
       \AttributeTok{x =} \StringTok{"Équipe"}\NormalTok{,}
       \AttributeTok{y =} \StringTok{"Ratio de buts par possession"}\NormalTok{) }\SpecialCharTok{+}
  \FunctionTok{theme}\NormalTok{(}\AttributeTok{axis.text.x =} \FunctionTok{element\_text}\NormalTok{(}\AttributeTok{angle =} \DecValTok{45}\NormalTok{, }\AttributeTok{hjust =} \DecValTok{1}\NormalTok{))}
\end{Highlighting}
\end{Shaded}

\includegraphics{rapport_files/figure-latex/unnamed-chunk-23-1.pdf}

\subsection{Graphique 3 : Top et Bottom 10 Équipes en Termes de
Possession en
2023}\label{graphique-3-top-et-bottom-10-uxe9quipes-en-termes-de-possession-en-2023}

Ce graphique en barres affiche le pourcentage de possession des 10
meilleures et 10 pires équipes, ainsi que leurs buts par match. Les 10
meilleures équipes en possession sont :

\begin{itemize}
\item
  \textbf{Manchester City (Premier League)}

  \begin{itemize}
  \tightlist
  \item
    Possession : 64.7\%
  \item
    Buts par match : 2.42
  \end{itemize}
\item
  \textbf{Barcelona (La Liga)}

  \begin{itemize}
  \tightlist
  \item
    Possession : 64.3\%
  \end{itemize}
\item
  Buts par match : 1.82
\item
  \textbf{Bayern Munich (Bundesliga)}

  \begin{itemize}
  \tightlist
  \item
    Possession : 64.2\%
  \item
    Buts par match : 2.37
  \end{itemize}
\item
  \textbf{Napoli (Serie A)}

  \begin{itemize}
  \tightlist
  \item
    Possession : 61.8\%
  \item
    Buts par match : 1.97
  \end{itemize}
\item
  \textbf{Real Madrid (La Liga)}

  \begin{itemize}
  \tightlist
  \item
    Possession : 61.0\%
  \item
    Buts par match : 1.92
  \end{itemize}
\item
  \textbf{Liverpool (Premier League)}

  \begin{itemize}
  \tightlist
  \item
    Possession : 60.8\%
  \item
    Buts par match : 1.87
  \end{itemize}
\item
  \textbf{Paris S-G (Ligue 1)}

  \begin{itemize}
  \tightlist
  \item
    Possession : 60.7\%
  \item
    Buts par match : 2.26
  \end{itemize}
\end{itemize}

\begin{Shaded}
\begin{Highlighting}[]
\NormalTok{top\_teams\_possession }\OtherTok{\textless{}{-}} \FunctionTok{data.frame}\NormalTok{(}
\NormalTok{  é}\AttributeTok{quipe =} \FunctionTok{c}\NormalTok{(}\StringTok{"Manchester City"}\NormalTok{, }\StringTok{"Barcelona"}\NormalTok{, }\StringTok{"Bayern Munich"}\NormalTok{, }\StringTok{"Napoli"}\NormalTok{, }\StringTok{"Real Madrid"}\NormalTok{, }\StringTok{"Liverpool"}\NormalTok{, }\StringTok{"Paris S{-}G"}\NormalTok{),}
  \AttributeTok{possession =} \FunctionTok{c}\NormalTok{(}\FloatTok{64.7}\NormalTok{, }\FloatTok{64.3}\NormalTok{, }\FloatTok{64.2}\NormalTok{, }\FloatTok{61.8}\NormalTok{, }\FloatTok{61.0}\NormalTok{, }\FloatTok{60.8}\NormalTok{, }\FloatTok{60.7}\NormalTok{),}
  \AttributeTok{buts\_par\_match =} \FunctionTok{c}\NormalTok{(}\FloatTok{2.42}\NormalTok{, }\FloatTok{1.82}\NormalTok{, }\FloatTok{2.37}\NormalTok{, }\FloatTok{1.97}\NormalTok{, }\FloatTok{1.92}\NormalTok{, }\FloatTok{1.87}\NormalTok{, }\FloatTok{2.26}\NormalTok{)}
\NormalTok{)}

\FunctionTok{ggplot}\NormalTok{(top\_teams\_possession, }\FunctionTok{aes}\NormalTok{(}\AttributeTok{x =} \FunctionTok{reorder}\NormalTok{(équipe, }\SpecialCharTok{{-}}\NormalTok{possession), }\AttributeTok{y =}\NormalTok{ possession, }\AttributeTok{fill =}\NormalTok{ équipe)) }\SpecialCharTok{+}
  \FunctionTok{geom\_bar}\NormalTok{(}\AttributeTok{stat =} \StringTok{"identity"}\NormalTok{) }\SpecialCharTok{+}
  \FunctionTok{geom\_text}\NormalTok{(}\FunctionTok{aes}\NormalTok{(}\AttributeTok{label =} \FunctionTok{paste}\NormalTok{(}\StringTok{"Possession: "}\NormalTok{, possession, }\StringTok{"\%}\SpecialCharTok{\textbackslash{}n}\StringTok{Buts/match: "}\NormalTok{, buts\_par\_match)), }
            \AttributeTok{vjust =} \SpecialCharTok{{-}}\FloatTok{0.5}\NormalTok{, }\AttributeTok{color =} \StringTok{"black"}\NormalTok{) }\SpecialCharTok{+}
  \FunctionTok{labs}\NormalTok{(}\AttributeTok{title =} \StringTok{"Top 10 équipes en termes de possession en 2023"}\NormalTok{,}
       \AttributeTok{x =} \StringTok{"Équipe"}\NormalTok{,}
       \AttributeTok{y =} \StringTok{"Possession (\%)"}\NormalTok{) }\SpecialCharTok{+}
  \FunctionTok{theme}\NormalTok{(}\AttributeTok{axis.text.x =} \FunctionTok{element\_text}\NormalTok{(}\AttributeTok{angle =} \DecValTok{45}\NormalTok{, }\AttributeTok{hjust =} \DecValTok{1}\NormalTok{))}
\end{Highlighting}
\end{Shaded}

\includegraphics{rapport_files/figure-latex/unnamed-chunk-24-1.pdf}

\subsection{Graphique 4 : Relation entre la Possession et les Buts
Encaissés}\label{graphique-4-relation-entre-la-possession-et-les-buts-encaissuxe9s}

Ce graphique en nuages de points montre la relation entre le pourcentage
de possession et les buts encaissés (buts pris).

\begin{Shaded}
\begin{Highlighting}[]
\CommentTok{\# Extraire les données des grandes ligues pour la saison 2023}
\NormalTok{big5\_stats }\OtherTok{\textless{}{-}} \FunctionTok{fb\_big5\_advanced\_season\_stats}\NormalTok{(}\AttributeTok{season\_end\_year =} \DecValTok{2023}\NormalTok{, }\AttributeTok{stat\_type =} \StringTok{"standard"}\NormalTok{, }\AttributeTok{team\_or\_player =} \StringTok{"team"}\NormalTok{)}
\NormalTok{big5\_possession }\OtherTok{\textless{}{-}} \FunctionTok{fb\_big5\_advanced\_season\_stats}\NormalTok{(}\AttributeTok{season\_end\_year =} \DecValTok{2023}\NormalTok{, }\AttributeTok{stat\_type =} \StringTok{"possession"}\NormalTok{, }\AttributeTok{team\_or\_player =} \StringTok{"team"}\NormalTok{)}

\CommentTok{\# Séparer les buts marqués et encaissés}
\NormalTok{buts\_marques }\OtherTok{\textless{}{-}}\NormalTok{ big5\_stats }\SpecialCharTok{\%\textgreater{}\%}
  \FunctionTok{filter}\NormalTok{(Team\_or\_Opponent }\SpecialCharTok{==} \StringTok{"team"}\NormalTok{) }\SpecialCharTok{\%\textgreater{}\%}
  \FunctionTok{select}\NormalTok{(Squad, Comp, Gls)}

\NormalTok{buts\_encaisses }\OtherTok{\textless{}{-}}\NormalTok{ big5\_stats }\SpecialCharTok{\%\textgreater{}\%}
  \FunctionTok{filter}\NormalTok{(Team\_or\_Opponent }\SpecialCharTok{==} \StringTok{"opponent"}\NormalTok{) }\SpecialCharTok{\%\textgreater{}\%}
  \FunctionTok{select}\NormalTok{(Squad, Comp, }\AttributeTok{GA =}\NormalTok{ Gls)  }\CommentTok{\# Renommer Gls en GA pour clarté}

\CommentTok{\# Fusionner les données de possession avec les données de buts marqués et encaissés}
\NormalTok{data }\OtherTok{\textless{}{-}}\NormalTok{ big5\_possession }\SpecialCharTok{\%\textgreater{}\%}
  \FunctionTok{inner\_join}\NormalTok{(buts\_marques, }\AttributeTok{by =} \FunctionTok{c}\NormalTok{(}\StringTok{"Squad"}\NormalTok{, }\StringTok{"Comp"}\NormalTok{)) }\SpecialCharTok{\%\textgreater{}\%}
  \FunctionTok{inner\_join}\NormalTok{(buts\_encaisses, }\AttributeTok{by =} \FunctionTok{c}\NormalTok{(}\StringTok{"Squad"}\NormalTok{, }\StringTok{"Comp"}\NormalTok{))}

\CommentTok{\# Visualisation de la corrélation entre la possession et les buts encaissés}
\FunctionTok{ggplot}\NormalTok{(data, }\FunctionTok{aes}\NormalTok{(}\AttributeTok{x =}\NormalTok{ Poss, }\AttributeTok{y =}\NormalTok{ GA)) }\SpecialCharTok{+}
  \FunctionTok{geom\_point}\NormalTok{() }\SpecialCharTok{+}
  \FunctionTok{geom\_smooth}\NormalTok{(}\AttributeTok{method =} \StringTok{"lm"}\NormalTok{, }\AttributeTok{se =} \ConstantTok{TRUE}\NormalTok{, }\AttributeTok{color =} \StringTok{"red"}\NormalTok{) }\SpecialCharTok{+}
  \FunctionTok{labs}\NormalTok{(}\AttributeTok{title =} \StringTok{"Relation entre la possession et les buts encaissés"}\NormalTok{,}
       \AttributeTok{x =} \StringTok{"Possession (\%)"}\NormalTok{,}
       \AttributeTok{y =} \StringTok{"Buts encaissés"}\NormalTok{) }\SpecialCharTok{+}
  \FunctionTok{theme\_minimal}\NormalTok{()}
\end{Highlighting}
\end{Shaded}

\begin{verbatim}
## `geom_smooth()` using formula = 'y ~ x'
\end{verbatim}

\includegraphics{rapport_files/figure-latex/unnamed-chunk-25-1.pdf}

\subsection{Graphique 5 : Relation entre la Possession et les Buts
Marqués}\label{graphique-5-relation-entre-la-possession-et-les-buts-marquuxe9s}

Ce graphique en nuages de points illustre la relation entre le
pourcentage de possession et les buts marqués (buts mis).

\begin{Shaded}
\begin{Highlighting}[]
\CommentTok{\# Visualisation de la corrélation entre la possession et les buts marqués}
\FunctionTok{ggplot}\NormalTok{(data, }\FunctionTok{aes}\NormalTok{(}\AttributeTok{x =}\NormalTok{ Poss, }\AttributeTok{y =}\NormalTok{ Gls)) }\SpecialCharTok{+}
  \FunctionTok{geom\_point}\NormalTok{() }\SpecialCharTok{+}
  \FunctionTok{geom\_smooth}\NormalTok{(}\AttributeTok{method =} \StringTok{"lm"}\NormalTok{, }\AttributeTok{se =} \ConstantTok{TRUE}\NormalTok{, }\AttributeTok{color =} \StringTok{"blue"}\NormalTok{) }\SpecialCharTok{+}
  \FunctionTok{labs}\NormalTok{(}\AttributeTok{title =} \StringTok{"Relation entre la possession et les buts marqués"}\NormalTok{,}
       \AttributeTok{x =} \StringTok{"Possession (\%)"}\NormalTok{,}
       \AttributeTok{y =} \StringTok{"Buts marqués"}\NormalTok{) }\SpecialCharTok{+}
  \FunctionTok{theme\_minimal}\NormalTok{()}
\end{Highlighting}
\end{Shaded}

\begin{verbatim}
## `geom_smooth()` using formula = 'y ~ x'
\end{verbatim}

\includegraphics{rapport_files/figure-latex/unnamed-chunk-26-1.pdf}

\section{Vérification des
Hypothèses}\label{vuxe9rification-des-hypothuxe8ses}

\subsection{Hypothèse 1 : Les équipes avec une plus grande possession de
balle ont tendance à marquer plus de buts par
match.}\label{hypothuxe8se-1-les-uxe9quipes-avec-une-plus-grande-possession-de-balle-ont-tendance-uxe0-marquer-plus-de-buts-par-match.}

\begin{itemize}
\tightlist
\item
  \textbf{Vérification :} Le graphique 5 confirme clairement cette
  hypothèse. La corrélation positive entre la possession et les buts
  marqués montre que les équipes qui dominent en possession marquent
  plus de buts.
\end{itemize}

\subsection{Hypothèse 2 : Les équipes qui dominent en possession de
balle sont plus efficaces dans la création d'occasions de
but.}\label{hypothuxe8se-2-les-uxe9quipes-qui-dominent-en-possession-de-balle-sont-plus-efficaces-dans-la-cruxe9ation-doccasions-de-but.}

\begin{itemize}
\tightlist
\item
  \textbf{Vérification :} Les graphiques 2 et 3 confirment cette
  hypothèse. La corrélation positive entre la possession et le nombre de
  buts par match montre que les équipes avec une plus grande possession
  créent plus d'occasions de but et sont plus précises dans leurs
  tentatives.
\end{itemize}

\subsection{Hypothèse 3 : La possession de balle est liée à la capacité
d'une équipe à contrôler le
jeu.}\label{hypothuxe8se-3-la-possession-de-balle-est-liuxe9e-uxe0-la-capacituxe9-dune-uxe9quipe-uxe0-contruxf4ler-le-jeu.}

\begin{itemize}
\tightlist
\item
  \textbf{Vérification :} Les graphiques 1 et 3 supportent cette
  hypothèse. Une plus grande possession de balle est associée à un
  meilleur contrôle du jeu, comme en témoigne la possession élevée et le
  nombre élevé de buts marqués par les meilleures équipes.
\end{itemize}

\section{Conclusion}\label{conclusion}

L'analyse des données et des graphiques montre que la possession de
balle est un facteur crucial pour la performance globale d'une équipe de
football. Les équipes qui dominent la possession de balle tendent à
marquer plus de buts et à en concéder moins, ce qui leur permet de mieux
contrôler le jeu et d'obtenir de meilleurs résultats. Le profil de
l'équipe parfaite en termes de possession de balle serait donc une
équipe qui maintient une possession élevée, réussit un grand nombre de
passes et crée de nombreuses occasions de but.

\subsubsection{\texorpdfstring{\textbf{Quelles sont les différences
entre les championnats et les coupes
?}}{Quelles sont les différences entre les championnats et les coupes ?}}\label{quelles-sont-les-diffuxe9rences-entre-les-championnats-et-les-coupes}

(traitée par Nassim)

Dans cette section, nous explorons les différences entre les
compétitions de championnat, comme la Premier League, et les
compétitions en coupe, telles que la FA Cup. Une des métriques que nous
examinons est le nombre moyen de buts marqués par match, ce qui pourrait
indiquer une approche stratégique différente due à la nature de chaque
type de compétition. On pourrait s'attendre à ce que les matchs de
coupe, qui sont souvent des affrontements à élimination directe,
encouragent un jeu plus agressif et potentiellement plus de buts, tandis
que les championnats pourraient favoriser une approche plus mesurée et
stratégique sur la durée d'une saison.

Traitement des données : Nous avons utilisé la bibliothèque
worldfootballR pour extraire les résultats des matchs des équipes
participant à la FA Cup et à la Premier League pour la saison 2022-2023.
Nous avons ensuite nettoyé et transformé ces données pour en faciliter
l'analyse. Voici le code utilisé pour obtenir ces données et calculer
les moyennes de buts marqués :

\begin{Shaded}
\begin{Highlighting}[]
\FunctionTok{library}\NormalTok{(worldfootballR)}
\FunctionTok{library}\NormalTok{(dplyr)}
\FunctionTok{library}\NormalTok{(ggplot2)}

\CommentTok{\# Extraction des URLs des équipes pour la FA Cup et la Premier League}
\NormalTok{fa\_cup\_url }\OtherTok{\textless{}{-}} \StringTok{"https://fbref.com/en/comps/514/2022{-}2023/2022{-}2023{-}FA{-}Cup{-}Stats"}
\NormalTok{fa\_cup\_team\_urls }\OtherTok{\textless{}{-}} \FunctionTok{fb\_teams\_urls}\NormalTok{(fa\_cup\_url)}
\NormalTok{pl\_team\_urls }\OtherTok{\textless{}{-}} \FunctionTok{fb\_teams\_urls}\NormalTok{(}\StringTok{"https://fbref.com/en/comps/9/2022{-}2023/2022{-}2023{-}Premier{-}League{-}Stats"}\NormalTok{)}

\CommentTok{\# Fonctions pour obtenir les résultats des matchs}
\NormalTok{get\_fa\_cup\_team\_results }\OtherTok{\textless{}{-}} \ControlFlowTok{function}\NormalTok{(team\_url) \{}
\NormalTok{  results }\OtherTok{\textless{}{-}} \FunctionTok{fb\_team\_match\_results}\NormalTok{(team\_url)}
\NormalTok{  results }\SpecialCharTok{\%\textgreater{}\%} \FunctionTok{mutate}\NormalTok{(}\AttributeTok{Team =} \FunctionTok{gsub}\NormalTok{(}\StringTok{".*/"}\NormalTok{, }\StringTok{""}\NormalTok{, team\_url), }\AttributeTok{Team =} \FunctionTok{gsub}\NormalTok{(}\StringTok{"{-}Stats"}\NormalTok{, }\StringTok{""}\NormalTok{, Team),}
                    \AttributeTok{Competition =} \StringTok{"FA Cup"}\NormalTok{, }\AttributeTok{GF =} \FunctionTok{as.numeric}\NormalTok{(GF), }\AttributeTok{GA =} \FunctionTok{as.numeric}\NormalTok{(GA)) }\SpecialCharTok{\%\textgreater{}\%}
    \FunctionTok{select}\NormalTok{(Team, Date, Opponent, Result, Venue, GF, GA, }\AttributeTok{Comp =}\NormalTok{ Competition)}
\NormalTok{\}}
\end{Highlighting}
\end{Shaded}

\subsubsection{Visualisation des moyennes de buts
marqués}\label{visualisation-des-moyennes-de-buts-marquuxe9s}

Nous avons créé un graphique pour visualiser les différences entre la
moyenne des buts marqués par match dans la FA Cup et dans la Premier
League :

\begin{Shaded}
\begin{Highlighting}[]
\FunctionTok{library}\NormalTok{(worldfootballR)}
\FunctionTok{library}\NormalTok{(dplyr)}
\FunctionTok{library}\NormalTok{(ggplot2)}

\CommentTok{\# URL pour la FA Cup 2022{-}2023}
\NormalTok{fa\_cup\_url }\OtherTok{\textless{}{-}} \StringTok{"https://fbref.com/en/comps/514/2022{-}2023/2022{-}2023{-}FA{-}Cup{-}Stats"}

\CommentTok{\# Extraire les URLs des équipes pour la FA Cup}
\NormalTok{fa\_cup\_team\_urls }\OtherTok{\textless{}{-}} \FunctionTok{fb\_teams\_urls}\NormalTok{(fa\_cup\_url)}

\CommentTok{\# Fonction pour obtenir les résultats des matchs d\textquotesingle{}une équipe dans la FA Cup}
\NormalTok{get\_fa\_cup\_team\_results }\OtherTok{\textless{}{-}} \ControlFlowTok{function}\NormalTok{(team\_url) \{}
\NormalTok{  results }\OtherTok{\textless{}{-}} \FunctionTok{fb\_team\_match\_results}\NormalTok{(team\_url)}
\NormalTok{  results }\OtherTok{\textless{}{-}}\NormalTok{ results }\SpecialCharTok{\%\textgreater{}\%}
    \FunctionTok{mutate}\NormalTok{(}\AttributeTok{Team =} \FunctionTok{gsub}\NormalTok{(}\StringTok{".*/"}\NormalTok{, }\StringTok{""}\NormalTok{, team\_url),  }\CommentTok{\# Extraire et nettoyer le nom de l\textquotesingle{}équipe de l\textquotesingle{}URL}
           \AttributeTok{Team =} \FunctionTok{gsub}\NormalTok{(}\StringTok{"{-}Stats"}\NormalTok{, }\StringTok{""}\NormalTok{, Team),}
           \AttributeTok{Competition =} \StringTok{"FA Cup"}\NormalTok{,}
           \AttributeTok{GF =} \FunctionTok{as.numeric}\NormalTok{(GF),  }\CommentTok{\# Assurez{-}vous que les buts marqués sont numériques}
           \AttributeTok{GA =} \FunctionTok{as.numeric}\NormalTok{(GA)) }\SpecialCharTok{\%\textgreater{}\%}
    \FunctionTok{select}\NormalTok{(Team, Date, Opponent, Result, Venue, GF, GA, }\AttributeTok{Comp =}\NormalTok{ Competition)}
  \FunctionTok{return}\NormalTok{(results)}
\NormalTok{\}}
\end{Highlighting}
\end{Shaded}

\subsubsection{\texorpdfstring{\textbf{Quelles sont les différences
entre les championnats et les coupes
?}}{Quelles sont les différences entre les championnats et les coupes ?}}\label{quelles-sont-les-diffuxe9rences-entre-les-championnats-et-les-coupes-1}

(traitée par Nassim)

\subsection{Hypothèse}\label{hypothuxe8se}

Notre hypothèse est que dans les coupes, où les matchs se jouent
généralement en élimination directe, les équipes adoptent une approche
plus prudente et tactique pour mieux gérer le match et minimiser les
risques. Cela peut se traduire par des performances différentes par
rapport aux championnats où les équipes ont une saison entière pour
s'adapter et se projeter. En particulier, nous nous attendons à ce qu'il
y ait moins de buts par match en coupe.

\section{Analyse des Données}\label{analyse-des-donnuxe9es}

\subsection{Moyenne des Buts par
Match}\label{moyenne-des-buts-par-match}

\begin{Shaded}
\begin{Highlighting}[]
\NormalTok{avg\_goals }\OtherTok{\textless{}{-}} \FunctionTok{data.frame}\NormalTok{(}
  \AttributeTok{Competition =} \FunctionTok{c}\NormalTok{(}\StringTok{"FA Cup"}\NormalTok{, }\StringTok{"Premier League"}\NormalTok{),}
  \AttributeTok{AvgGoals =} \FunctionTok{c}\NormalTok{(}\FloatTok{1.77}\NormalTok{, }\FloatTok{1.37}\NormalTok{)}
\NormalTok{)}

\FunctionTok{ggplot}\NormalTok{(avg\_goals, }\FunctionTok{aes}\NormalTok{(}\AttributeTok{x =}\NormalTok{ Competition, }\AttributeTok{y =}\NormalTok{ AvgGoals, }\AttributeTok{fill =}\NormalTok{ Competition)) }\SpecialCharTok{+}
  \FunctionTok{geom\_bar}\NormalTok{(}\AttributeTok{stat =} \StringTok{"identity"}\NormalTok{) }\SpecialCharTok{+}
  \FunctionTok{geom\_text}\NormalTok{(}\FunctionTok{aes}\NormalTok{(}\AttributeTok{label =} \FunctionTok{paste0}\NormalTok{(AvgGoals, }\StringTok{" buts/match"}\NormalTok{)), }\AttributeTok{vjust =} \SpecialCharTok{{-}}\FloatTok{0.5}\NormalTok{) }\SpecialCharTok{+}
  \FunctionTok{labs}\NormalTok{(}\AttributeTok{title =} \StringTok{"Nombre moyen de buts par match selon la compétition"}\NormalTok{,}
       \AttributeTok{x =} \StringTok{"Compétition"}\NormalTok{, }\AttributeTok{y =} \StringTok{"Nombre moyen de buts par match"}\NormalTok{) }\SpecialCharTok{+}
  \FunctionTok{scale\_fill\_manual}\NormalTok{(}\AttributeTok{values =} \FunctionTok{c}\NormalTok{(}\StringTok{"FA Cup"} \OtherTok{=} \StringTok{"blue"}\NormalTok{, }\StringTok{"Premier League"} \OtherTok{=} \StringTok{"red"}\NormalTok{)) }\SpecialCharTok{+}
  \FunctionTok{theme\_minimal}\NormalTok{()}
\end{Highlighting}
\end{Shaded}

\includegraphics{rapport_files/figure-latex/unnamed-chunk-29-1.pdf}

\subsubsection{Analyse des Résultats}\label{analyse-des-ruxe9sultats}

Les données montrent que la FA Cup a une moyenne de 1,77 buts par match
contre 1,37 pour la Premier League. Ce résultat est surprenant et
contredit notre hypothèse initiale selon laquelle il y aurait moins de
buts en coupe.

\paragraph{Hypothèses et
Explications}\label{hypothuxe8ses-et-explications}

\begin{enumerate}
\def\labelenumi{\arabic{enumi}.}
\tightlist
\item
  \textbf{Nature des Adversaires} : En FA Cup, les équipes de Premier
  League peuvent rencontrer des équipes de divisions inférieures, ce qui
  pourrait conduire à un nombre plus élevé de buts en raison de la
  différence de niveau.
\item
  \textbf{Approche Offensif} : Les équipes peuvent adopter une approche
  plus offensive en FA Cup pour éviter les prolongations et les tirs au
  but, ce qui peut résulter en plus de buts marqués.
\item
  \textbf{Moins de Pression} : Les équipes de Premier League peuvent
  jouer avec moins de pression en FA Cup, sachant que cette compétition
  n'affecte pas leur classement en championnat, ce qui pourrait libérer
  les attaquants.
\end{enumerate}

\subsection{Pourcentage d'Arrêts}\label{pourcentage-darruxeats}

\begin{Shaded}
\begin{Highlighting}[]
\NormalTok{avg\_saves }\OtherTok{\textless{}{-}} \FunctionTok{data.frame}\NormalTok{(}
  \AttributeTok{Competition =} \FunctionTok{c}\NormalTok{(}\StringTok{"FA Cup"}\NormalTok{, }\StringTok{"Premier League"}\NormalTok{),}
  \AttributeTok{AvgSavePercentage =} \FunctionTok{c}\NormalTok{(}\FloatTok{61.68}\NormalTok{, }\FloatTok{66.23}\NormalTok{)}
\NormalTok{)}

\FunctionTok{ggplot}\NormalTok{(avg\_saves, }\FunctionTok{aes}\NormalTok{(}\AttributeTok{x =}\NormalTok{ Competition, }\AttributeTok{y =}\NormalTok{ AvgSavePercentage, }\AttributeTok{fill =}\NormalTok{ Competition)) }\SpecialCharTok{+}
  \FunctionTok{geom\_bar}\NormalTok{(}\AttributeTok{stat =} \StringTok{"identity"}\NormalTok{) }\SpecialCharTok{+}
  \FunctionTok{geom\_text}\NormalTok{(}\FunctionTok{aes}\NormalTok{(}\AttributeTok{label =} \FunctionTok{paste0}\NormalTok{(AvgSavePercentage, }\StringTok{" \%"}\NormalTok{)), }\AttributeTok{vjust =} \SpecialCharTok{{-}}\FloatTok{0.5}\NormalTok{) }\SpecialCharTok{+}
  \FunctionTok{labs}\NormalTok{(}\AttributeTok{title =} \StringTok{"Pourcentage d\textquotesingle{}arrêts selon la compétition"}\NormalTok{,}
       \AttributeTok{x =} \StringTok{"Compétition"}\NormalTok{, }\AttributeTok{y =} \StringTok{"Pourcentage d\textquotesingle{}arrêts"}\NormalTok{) }\SpecialCharTok{+}
  \FunctionTok{scale\_fill\_manual}\NormalTok{(}\AttributeTok{values =} \FunctionTok{c}\NormalTok{(}\StringTok{"FA Cup"} \OtherTok{=} \StringTok{"blue"}\NormalTok{, }\StringTok{"Premier League"} \OtherTok{=} \StringTok{"red"}\NormalTok{)) }\SpecialCharTok{+}
  \FunctionTok{theme\_minimal}\NormalTok{()}
\end{Highlighting}
\end{Shaded}

\includegraphics{rapport_files/figure-latex/unnamed-chunk-30-1.pdf}

\subsubsection{Analyse des Résultats}\label{analyse-des-ruxe9sultats-1}

Le pourcentage d'arrêts est légèrement supérieur en Premier League
(66,23 \%) par rapport à la FA Cup (61,68 \%). Cette différence peut
être significative lorsqu'elle est examinée en parallèle avec le nombre
de buts concédés par match dans chaque compétition. En Premier League,
où les équipes ont un pourcentage d'arrêts plus élevé, le nombre moyen
de buts concédés par match est en moyenne plus bas (1,37) par rapport à
la FA Cup (1,77).Cela suggère que les équipes de Premier League ont une
meilleure efficacité défensive, ce qui se traduit par un pourcentage
d'arrêts plus élevé et un nombre de buts concédés plus faible par match.
En revanche, en FA Cup, où le pourcentage d'arrêts est légèrement
inférieur, les équipes ont tendance à concéder plus de buts par match.
\#\#\#\# Hypothèses et Explications

\begin{enumerate}
\def\labelenumi{\arabic{enumi}.}
\tightlist
\item
  \textbf{Niveau de Compétition} : En Premier League, les gardiens
  affrontent régulièrement des attaquants de haut niveau, ce qui peut
  les rendre plus performants sur la durée.
\item
  \textbf{Rotation des Gardiens} : En FA Cup, certaines équipes
  utilisent leurs gardiens remplaçants, ce qui pourrait expliquer un
  pourcentage d'arrêts légèrement inférieur.
\item
  \textbf{Style de Jeu} : Les matchs de coupe peuvent être plus ouverts
  avec des attaques fréquentes, ce qui peut rendre le travail des
  gardiens plus difficile.
\end{enumerate}

\subsection{Pourcentage de Clean
Sheets}\label{pourcentage-de-clean-sheets}

\begin{Shaded}
\begin{Highlighting}[]
\NormalTok{avg\_cleansheet }\OtherTok{\textless{}{-}} \FunctionTok{data.frame}\NormalTok{(}
  \AttributeTok{Competition =} \FunctionTok{c}\NormalTok{(}\StringTok{"FA Cup"}\NormalTok{, }\StringTok{"Premier League"}\NormalTok{),}
  \AttributeTok{AvgCleanSheetPercentage =} \FunctionTok{c}\NormalTok{(}\FloatTok{20.66}\NormalTok{, }\FloatTok{30.74}\NormalTok{)}
\NormalTok{)}

\FunctionTok{ggplot}\NormalTok{(avg\_cleansheet, }\FunctionTok{aes}\NormalTok{(}\AttributeTok{x =}\NormalTok{ Competition, }\AttributeTok{y =}\NormalTok{ AvgCleanSheetPercentage, }\AttributeTok{fill =}\NormalTok{ Competition)) }\SpecialCharTok{+}
  \FunctionTok{geom\_bar}\NormalTok{(}\AttributeTok{stat =} \StringTok{"identity"}\NormalTok{) }\SpecialCharTok{+}
  \FunctionTok{geom\_text}\NormalTok{(}\FunctionTok{aes}\NormalTok{(}\AttributeTok{label =} \FunctionTok{paste0}\NormalTok{(AvgCleanSheetPercentage, }\StringTok{" \%"}\NormalTok{)), }\AttributeTok{vjust =} \SpecialCharTok{{-}}\FloatTok{0.5}\NormalTok{) }\SpecialCharTok{+}
  \FunctionTok{labs}\NormalTok{(}\AttributeTok{title =} \StringTok{"Pourcentage de clean sheets selon la compétition"}\NormalTok{,}
       \AttributeTok{x =} \StringTok{"Compétition"}\NormalTok{, }\AttributeTok{y =} \StringTok{"Pourcentage de clean sheets"}\NormalTok{) }\SpecialCharTok{+}
  \FunctionTok{scale\_fill\_manual}\NormalTok{(}\AttributeTok{values =} \FunctionTok{c}\NormalTok{(}\StringTok{"FA Cup"} \OtherTok{=} \StringTok{"blue"}\NormalTok{, }\StringTok{"Premier League"} \OtherTok{=} \StringTok{"red"}\NormalTok{)) }\SpecialCharTok{+}
  \FunctionTok{theme\_minimal}\NormalTok{()}
\end{Highlighting}
\end{Shaded}

\includegraphics{rapport_files/figure-latex/unnamed-chunk-31-1.pdf}

\subsubsection{Analyse des Résultats}\label{analyse-des-ruxe9sultats-2}

En Premier League, le pourcentage de clean sheets est de 30,74 \%,
tandis qu'en FA Cup, il est de 20,66 \%. Cette différence indique que
les équipes ont tendance à garder leur cage inviolée plus souvent en
Premier League qu'en FA Cup.

Cette observation est cohérente avec le nombre moyen de buts par match
dans chaque compétition. En Premier League, où les équipes ont un
pourcentage de clean sheets plus élevé, le nombre moyen de buts concédés
par match est en moyenne plus bas (1,37) par rapport à la FA Cup (1,77).
Cela suggère que les équipes de Premier League ont une meilleure
efficacité défensive, ce qui se traduit par un nombre de buts concédés
plus faible et donc un pourcentage de clean sheets plus élevé.

\paragraph{Hypothèses et
Explications}\label{hypothuxe8ses-et-explications-1}

\begin{enumerate}
\def\labelenumi{\arabic{enumi}.}
\tightlist
\item
  \textbf{Importance des Matchs} : En Premier League, chaque match
  compte pour le classement final, ce qui pousse les équipes à être plus
  défensives pour éviter les buts.
\item
  \textbf{Adversaires Variés en Coupe} : En FA Cup, les équipes de
  Premier League affrontent parfois des équipes de divisions inférieures
  qui peuvent adopter des tactiques très offensives dans l'espoir de
  créer une surprise.
\item
  \textbf{Gestion de l'Équipe} : Les équipes de Premier League peuvent
  être plus conservatrices dans leur gestion défensive au long de la
  saison, tandis qu'en coupe, elles peuvent prendre plus de risques.
\end{enumerate}

\subsection{Performance des Équipes
Sélectionnées}\label{performance-des-uxe9quipes-suxe9lectionnuxe9es}

\subsubsection{Nombre Moyen de Buts par
Match}\label{nombre-moyen-de-buts-par-match}

\begin{Shaded}
\begin{Highlighting}[]
\NormalTok{team\_goals }\OtherTok{\textless{}{-}} \FunctionTok{data.frame}\NormalTok{(}
  \AttributeTok{Team =} \FunctionTok{rep}\NormalTok{(}\FunctionTok{c}\NormalTok{(}\StringTok{"Fulham"}\NormalTok{, }\StringTok{"Manchester City"}\NormalTok{, }\StringTok{"Southampton"}\NormalTok{), }\AttributeTok{each =} \DecValTok{2}\NormalTok{),}
  \AttributeTok{Competition =} \FunctionTok{rep}\NormalTok{(}\FunctionTok{c}\NormalTok{(}\StringTok{"FA Cup"}\NormalTok{, }\StringTok{"Premier League"}\NormalTok{), }\DecValTok{3}\NormalTok{),}
  \AttributeTok{AvgGoals =} \FunctionTok{c}\NormalTok{(}\FloatTok{1.8}\NormalTok{, }\FloatTok{1.37}\NormalTok{, }\FloatTok{3.17}\NormalTok{, }\FloatTok{2.42}\NormalTok{, }\FloatTok{1.67}\NormalTok{, }\FloatTok{0.95}\NormalTok{)}
\NormalTok{)}

\FunctionTok{ggplot}\NormalTok{(team\_goals, }\FunctionTok{aes}\NormalTok{(}\AttributeTok{x =}\NormalTok{ Competition, }\AttributeTok{y =}\NormalTok{ AvgGoals, }\AttributeTok{fill =}\NormalTok{ Team)) }\SpecialCharTok{+}
  \FunctionTok{geom\_bar}\NormalTok{(}\AttributeTok{stat =} \StringTok{"identity"}\NormalTok{, }\AttributeTok{position =} \StringTok{"dodge"}\NormalTok{) }\SpecialCharTok{+}
  \FunctionTok{geom\_text}\NormalTok{(}\FunctionTok{aes}\NormalTok{(}\AttributeTok{label =} \FunctionTok{paste0}\NormalTok{(AvgGoals, }\StringTok{" buts/match"}\NormalTok{)), }\AttributeTok{position =} \FunctionTok{position\_dodge}\NormalTok{(}\AttributeTok{width =} \FloatTok{0.9}\NormalTok{), }\AttributeTok{vjust =} \SpecialCharTok{{-}}\FloatTok{0.5}\NormalTok{) }\SpecialCharTok{+}
  \FunctionTok{labs}\NormalTok{(}\AttributeTok{title =} \StringTok{"Nombre moyen de buts par match pour Fulham, Manchester City et Southampton selon la compétition"}\NormalTok{,}
       \AttributeTok{x =} \StringTok{"Compétition"}\NormalTok{, }\AttributeTok{y =} \StringTok{"Nombre moyen de buts par match"}\NormalTok{) }\SpecialCharTok{+}
  \FunctionTok{scale\_fill\_manual}\NormalTok{(}\AttributeTok{values =} \FunctionTok{c}\NormalTok{(}\StringTok{"Fulham"} \OtherTok{=} \StringTok{"black"}\NormalTok{, }\StringTok{"Manchester City"} \OtherTok{=} \StringTok{"skyblue"}\NormalTok{, }\StringTok{"Southampton"} \OtherTok{=} \StringTok{"red"}\NormalTok{)) }\SpecialCharTok{+}
  \FunctionTok{theme\_minimal}\NormalTok{()}
\end{Highlighting}
\end{Shaded}

\includegraphics{rapport_files/figure-latex/unnamed-chunk-32-1.pdf}

\subsubsection{Analyse des Résultats}\label{analyse-des-ruxe9sultats-3}

Les résultats montrent des variations significatives entre les
compétitions pour les équipes sélectionnées. Par exemple, Manchester
City marque en moyenne 3,17 buts par match en FA Cup contre 2,42 en
Premier League.

\paragraph{Hypothèses et
Explications}\label{hypothuxe8ses-et-explications-2}

\begin{enumerate}
\def\labelenumi{\arabic{enumi}.}
\tightlist
\item
  \textbf{Différences de Niveau des Adversaires} : Manchester City peut
  affronter des équipes plus faibles en FA Cup, ce qui se traduit par un
  plus grand nombre de buts.
\item
  \textbf{Motivation en Coupe} : Les équipes de bas de tableau comme
  Southampton peuvent adopter une stratégie plus offensive en FA Cup
  pour essayer de compenser leur manque de succès en championnat.
\item
  \textbf{Tactiques et Sélection des Joueurs} : Les équipes peuvent
  expérimenter différentes formations et donner du temps de jeu à des
  attaquants supplémentaires en FA Cup.
\end{enumerate}

\subsubsection{Pourcentage d'Arrêts}\label{pourcentage-darruxeats-1}

\begin{Shaded}
\begin{Highlighting}[]
\NormalTok{team\_saves }\OtherTok{\textless{}{-}} \FunctionTok{data.frame}\NormalTok{(}
  \AttributeTok{Team =} \FunctionTok{rep}\NormalTok{(}\FunctionTok{c}\NormalTok{(}\StringTok{"Fulham"}\NormalTok{, }\StringTok{"Manchester City"}\NormalTok{, }\StringTok{"Southampton"}\NormalTok{), }\AttributeTok{each =} \DecValTok{2}\NormalTok{),}
  \AttributeTok{Competition =} \FunctionTok{rep}\NormalTok{(}\FunctionTok{c}\NormalTok{(}\StringTok{"FA Cup"}\NormalTok{, }\StringTok{"Premier League"}\NormalTok{), }\DecValTok{3}\NormalTok{),}
  \AttributeTok{SavePercentage =} \FunctionTok{c}\NormalTok{(}\FloatTok{80.8}\NormalTok{, }\FloatTok{76.5}\NormalTok{, }\DecValTok{100}\NormalTok{, }\FloatTok{67.4}\NormalTok{, }\DecValTok{75}\NormalTok{, }\FloatTok{52.7}\NormalTok{)}
\NormalTok{)}

\FunctionTok{ggplot}\NormalTok{(team\_saves, }\FunctionTok{aes}\NormalTok{(}\AttributeTok{x =}\NormalTok{ Competition, }\AttributeTok{y =}\NormalTok{ SavePercentage, }\AttributeTok{fill =}\NormalTok{ Team)) }\SpecialCharTok{+}
  \FunctionTok{geom\_bar}\NormalTok{(}\AttributeTok{stat =} \StringTok{"identity"}\NormalTok{, }\AttributeTok{position =} \StringTok{"dodge"}\NormalTok{) }\SpecialCharTok{+}
  \FunctionTok{geom\_text}\NormalTok{(}\FunctionTok{aes}\NormalTok{(}\AttributeTok{label =} \FunctionTok{paste0}\NormalTok{(SavePercentage, }\StringTok{" \%"}\NormalTok{)), }\AttributeTok{position =} \FunctionTok{position\_dodge}\NormalTok{(}\AttributeTok{width =} \FloatTok{0.9}\NormalTok{), }\AttributeTok{vjust =} \SpecialCharTok{{-}}\FloatTok{0.5}\NormalTok{) }\SpecialCharTok{+}
  \FunctionTok{labs}\NormalTok{(}\AttributeTok{title =} \StringTok{"Pourcentage d\textquotesingle{}arrêts pour Fulham, Manchester City et Southampton selon la compétition"}\NormalTok{,}
       \AttributeTok{x =} \StringTok{"Compétition"}\NormalTok{, }\AttributeTok{y =} \StringTok{"Pourcentage d\textquotesingle{}arrêts"}\NormalTok{) }\SpecialCharTok{+}
  \FunctionTok{scale\_fill\_manual}\NormalTok{(}\AttributeTok{values =} \FunctionTok{c}\NormalTok{(}\StringTok{"Fulham"} \OtherTok{=} \StringTok{"black"}\NormalTok{, }\StringTok{"Manchester City"} \OtherTok{=} \StringTok{"skyblue"}\NormalTok{, }\StringTok{"Southampton"} \OtherTok{=} \StringTok{"red"}\NormalTok{)) }\SpecialCharTok{+}
  \FunctionTok{theme\_minimal}\NormalTok{()}
\end{Highlighting}
\end{Shaded}

\includegraphics{rapport_files/figure-latex/unnamed-chunk-33-1.pdf}

\subsubsection{Analyse des Résultats}\label{analyse-des-ruxe9sultats-4}

Les pourcentages d'arrêts varient également significativement entre les
compétitions pour les équipes sélectionnées. Par exemple, Manchester
City affiche un pourcentage d'arrêts de 100 \% en FA Cup, contre 67,4 \%
en Premier League. Cette différence est intéressante à examiner en
relation avec le nombre de buts marqués par Manchester City dans chaque
compétition. En FA Cup, où Manchester City affiche un pourcentage
d'arrêts de 100 \%, ils ont également marqué en moyenne 3,17 buts par
match. En revanche, en Premier League, où leur pourcentage d'arrêts est
de 67,4 \%, ils ont marqué en moyenne 2,42 buts par match. Cette
corrélation suggère que lorsque Manchester City est plus efficace dans
ses arrêts en FA Cup, cela peut potentiellement libérer leur potentiel
offensif, conduisant ainsi à un plus grand nombre de buts marqués par
match. En revanche, une efficacité réduite dans les arrêts en Premier
League peut être associée à une pression défensive accrue, ce qui peut
affecter leurs performances offensives.

\paragraph{Hypothèses et
Explications}\label{hypothuxe8ses-et-explications-3}

\begin{enumerate}
\def\labelenumi{\arabic{enumi}.}
\tightlist
\item
  \textbf{Rotation des Gardiens} : Les équipes peuvent utiliser des
  gardiens remplaçants en FA Cup, ce qui peut entraîner des performances
  variables.
\item
  \textbf{Approche Tactique} : Les équipes peuvent adapter leur
  stratégie en fonction du niveau de l'adversaire et de l'importance du
  match, ce qui peut influencer le pourcentage d'arrêts.
\item
  \textbf{Fatigue et Concentration} : La fatigue accumulée au cours de
  la saison peut affecter la concentration des gardiens en championnat,
  ce qui peut se traduire par un pourcentage d'arrêts plus faible.
\end{enumerate}

\subsubsection{Pourcentage de Clean
Sheets}\label{pourcentage-de-clean-sheets-1}

\begin{Shaded}
\begin{Highlighting}[]
\NormalTok{team\_cleansheets }\OtherTok{\textless{}{-}} \FunctionTok{data.frame}\NormalTok{(}
  \AttributeTok{Team =} \FunctionTok{rep}\NormalTok{(}\FunctionTok{c}\NormalTok{(}\StringTok{"Fulham"}\NormalTok{, }\StringTok{"Manchester City"}\NormalTok{, }\StringTok{"Southampton"}\NormalTok{), }\AttributeTok{each =} \DecValTok{2}\NormalTok{),}
  \AttributeTok{Competition =} \FunctionTok{rep}\NormalTok{(}\FunctionTok{c}\NormalTok{(}\StringTok{"FA Cup"}\NormalTok{, }\StringTok{"Premier League"}\NormalTok{), }\DecValTok{3}\NormalTok{),}
  \AttributeTok{CleanSheetPercentage =} \FunctionTok{c}\NormalTok{(}\FloatTok{13.5}\NormalTok{, }\FloatTok{11.7}\NormalTok{, }\DecValTok{13}\NormalTok{, }\FloatTok{15.53}\NormalTok{, }\DecValTok{14}\NormalTok{, }\FloatTok{10.87}\NormalTok{)}
\NormalTok{)}

\FunctionTok{ggplot}\NormalTok{(team\_cleansheets, }\FunctionTok{aes}\NormalTok{(}\AttributeTok{x =}\NormalTok{ Competition, }\AttributeTok{y =}\NormalTok{ CleanSheetPercentage, }\AttributeTok{fill =}\NormalTok{ Team)) }\SpecialCharTok{+}
  \FunctionTok{geom\_bar}\NormalTok{(}\AttributeTok{stat =} \StringTok{"identity"}\NormalTok{, }\AttributeTok{position =} \StringTok{"dodge"}\NormalTok{) }\SpecialCharTok{+}
  \FunctionTok{geom\_text}\NormalTok{(}\FunctionTok{aes}\NormalTok{(}\AttributeTok{label =} \FunctionTok{paste0}\NormalTok{(CleanSheetPercentage, }\StringTok{" \%"}\NormalTok{)), }\AttributeTok{position =} \FunctionTok{position\_dodge}\NormalTok{(}\AttributeTok{width =} \FloatTok{0.9}\NormalTok{), }\AttributeTok{vjust =} \SpecialCharTok{{-}}\FloatTok{0.5}\NormalTok{) }\SpecialCharTok{+}
  \FunctionTok{labs}\NormalTok{(}\AttributeTok{title =} \StringTok{"Pourcentage de clean sheets pour Fulham, Manchester City et Southampton selon la compétition"}\NormalTok{,}
       \AttributeTok{x =} \StringTok{"Compétition"}\NormalTok{, }\AttributeTok{y =} \StringTok{"Pourcentage de clean sheets"}\NormalTok{) }\SpecialCharTok{+}
  \FunctionTok{scale\_fill\_manual}\NormalTok{(}\AttributeTok{values =} \FunctionTok{c}\NormalTok{(}\StringTok{"Fulham"} \OtherTok{=} \StringTok{"black"}\NormalTok{, }\StringTok{"Manchester City"} \OtherTok{=} \StringTok{"skyblue"}\NormalTok{, }\StringTok{"Southampton"} \OtherTok{=} \StringTok{"red"}\NormalTok{)) }\SpecialCharTok{+}
  \FunctionTok{theme\_minimal}\NormalTok{()}
\end{Highlighting}
\end{Shaded}

\includegraphics{rapport_files/figure-latex/unnamed-chunk-34-1.pdf}

\subsubsection{Nombre Moyen de Tirs par
Match}\label{nombre-moyen-de-tirs-par-match}

En moyenne, les équipes ont pris plus de tirs par match en FA Cup (13,5)
que en Premier League (11,7), ce qui correspond également à un nombre de
tirs cadrés par match plus élevé en FA Cup (4,4) par rapport à la
Premier League (4,1). Cette tendance suggère une approche plus offensive
adoptée en coupe, où les équipes cherchent à marquer rapidement pour
éviter les prolongations ou les tirs au but. La différence dans le
nombre de tirs peut également refléter la variabilité du niveau des
adversaires en FA Cup, où les équipes de divisions inférieures peuvent
être plus ouvertes en attaque, offrant ainsi plus d'opportunités de tirs
aux équipes de Premier League.En moyenne, les équipes ont marqué plus de
buts par match en FA Cup (1,77) que en Premier League (1,37), ce qui
suggère une tendance offensive plus prononcée dans cette compétition.
Cette observation est cohérente avec le fait que les équipes en FA Cup
prennent également plus de tirs par match, ce qui indique une approche
plus offensive et une volonté de marquer des buts.

\begin{Shaded}
\begin{Highlighting}[]
\NormalTok{avg\_shots }\OtherTok{\textless{}{-}} \FunctionTok{data.frame}\NormalTok{(}
  \AttributeTok{Competition =} \FunctionTok{c}\NormalTok{(}\StringTok{"FA Cup"}\NormalTok{, }\StringTok{"Premier League"}\NormalTok{),}
  \AttributeTok{AvgShots =} \FunctionTok{c}\NormalTok{(}\FloatTok{13.5}\NormalTok{, }\FloatTok{11.7}\NormalTok{)}
\NormalTok{)}

\FunctionTok{ggplot}\NormalTok{(avg\_shots, }\FunctionTok{aes}\NormalTok{(}\AttributeTok{x =}\NormalTok{ Competition, }\AttributeTok{y =}\NormalTok{ AvgShots, }\AttributeTok{fill =}\NormalTok{ Competition)) }\SpecialCharTok{+}
  \FunctionTok{geom\_bar}\NormalTok{(}\AttributeTok{stat =} \StringTok{"identity"}\NormalTok{) }\SpecialCharTok{+}
  \FunctionTok{geom\_text}\NormalTok{(}\FunctionTok{aes}\NormalTok{(}\AttributeTok{label =}\NormalTok{ AvgShots), }\AttributeTok{vjust =} \SpecialCharTok{{-}}\FloatTok{0.5}\NormalTok{) }\SpecialCharTok{+}
  \FunctionTok{labs}\NormalTok{(}\AttributeTok{title =} \StringTok{"Nombre moyen de tirs par match selon la compétition"}\NormalTok{,}
       \AttributeTok{x =} \StringTok{"Compétition"}\NormalTok{, }\AttributeTok{y =} \StringTok{"Nombre moyen de tirs par match"}\NormalTok{) }\SpecialCharTok{+}
  \FunctionTok{scale\_fill\_manual}\NormalTok{(}\AttributeTok{values =} \FunctionTok{c}\NormalTok{(}\StringTok{"FA Cup"} \OtherTok{=} \StringTok{"blue"}\NormalTok{, }\StringTok{"Premier League"} \OtherTok{=} \StringTok{"red"}\NormalTok{)) }\SpecialCharTok{+}
  \FunctionTok{theme\_minimal}\NormalTok{()}
\end{Highlighting}
\end{Shaded}

\includegraphics{rapport_files/figure-latex/unnamed-chunk-35-1.pdf}

\subsubsection{Nombre Moyen de Tirs Cadrés par
Match}\label{nombre-moyen-de-tirs-cadruxe9s-par-match}

Le nombre moyen de tirs cadrés par match est également plus élevé en FA
Cup (4,4) par rapport à la Premier League (4,1). Cette tendance est
intéressante à observer, d'autant plus que le nombre moyen de buts par
match est également plus élevé en FA Cup (1,77) par rapport à la Premier
League (1,37). Cela suggère que les équipes en coupe ont non seulement
tendance à cibler plus souvent le but adverse avec leurs tirs, mais
aussi qu'elles réussissent à convertir ces occasions en buts à un taux
légèrement plus élevé que lors des matchs de championnat. Ces
observations soulignent l'importance de la précision des tirs et de
l'efficacité offensive dans les matchs de coupe, où chaque occasion peut
avoir un impact significatif sur le résultat final.

\begin{Shaded}
\begin{Highlighting}[]
\NormalTok{avg\_shots\_on\_target }\OtherTok{\textless{}{-}} \FunctionTok{data.frame}\NormalTok{(}
  \AttributeTok{Competition =} \FunctionTok{c}\NormalTok{(}\StringTok{"FA Cup"}\NormalTok{, }\StringTok{"Premier League"}\NormalTok{),}
  \AttributeTok{AvgShotsOnTarget =} \FunctionTok{c}\NormalTok{(}\FloatTok{4.4}\NormalTok{, }\FloatTok{4.1}\NormalTok{)}
\NormalTok{)}

\FunctionTok{ggplot}\NormalTok{(avg\_shots\_on\_target, }\FunctionTok{aes}\NormalTok{(}\AttributeTok{x =}\NormalTok{ Competition, }\AttributeTok{y =}\NormalTok{ AvgShotsOnTarget, }\AttributeTok{fill =}\NormalTok{ Competition)) }\SpecialCharTok{+}
  \FunctionTok{geom\_bar}\NormalTok{(}\AttributeTok{stat =} \StringTok{"identity"}\NormalTok{) }\SpecialCharTok{+}
  \FunctionTok{geom\_text}\NormalTok{(}\FunctionTok{aes}\NormalTok{(}\AttributeTok{label =}\NormalTok{ AvgShotsOnTarget), }\AttributeTok{vjust =} \SpecialCharTok{{-}}\FloatTok{0.5}\NormalTok{) }\SpecialCharTok{+}
  \FunctionTok{labs}\NormalTok{(}\AttributeTok{title =} \StringTok{"Nombre moyen de tirs cadrés par match selon la compétition"}\NormalTok{,}
       \AttributeTok{x =} \StringTok{"Compétition"}\NormalTok{, }\AttributeTok{y =} \StringTok{"Nombre moyen de tirs cadrés par match"}\NormalTok{) }\SpecialCharTok{+}
  \FunctionTok{scale\_fill\_manual}\NormalTok{(}\AttributeTok{values =} \FunctionTok{c}\NormalTok{(}\StringTok{"FA Cup"} \OtherTok{=} \StringTok{"blue"}\NormalTok{, }\StringTok{"Premier League"} \OtherTok{=} \StringTok{"red"}\NormalTok{)) }\SpecialCharTok{+}
  \FunctionTok{theme\_minimal}\NormalTok{()}
\end{Highlighting}
\end{Shaded}

\includegraphics{rapport_files/figure-latex/unnamed-chunk-36-1.pdf}

\subsubsection{Conclusion}\label{conclusion-1}

En incluant le nombre moyen de tirs cadrés par match dans notre analyse,
nous avons observé une corrélation plus forte entre les tirs, les tirs
cadrés et les arrêts des gardiens. Les équipes qui prennent plus de tirs
et les ciblent plus souvent parviennent à mettre davantage de pression
sur les gardiens adverses, ce qui réduit leur efficacité globale. Ces
observations soulignent l'importance de la qualité des tirs et de la
défense dans le déroulement des matchs de football, et soulignent la
nécessité pour les équipes de trouver un équilibre entre l'attaque et la
défense pour réussir dans différentes compétitions.

\subsubsection{Ratio Buts/Tirs}\label{ratio-butstirs}

Le ratio buts/tirs est un indicateur important de l'efficacité des
équipes à convertir leurs occasions en buts. En comparant la FA Cup et
la Premier League, nous pouvons observer comment ce ratio varie selon la
compétition.

\begin{Shaded}
\begin{Highlighting}[]
\NormalTok{goal\_to\_shot\_ratio }\OtherTok{\textless{}{-}} \FunctionTok{data.frame}\NormalTok{(}
  \AttributeTok{Competition =} \FunctionTok{c}\NormalTok{(}\StringTok{"FA Cup"}\NormalTok{, }\StringTok{"Premier League"}\NormalTok{),}
  \AttributeTok{GoalToShotRatio =} \FunctionTok{c}\NormalTok{(}\FloatTok{0.105}\NormalTok{, }\FloatTok{0.1}\NormalTok{)}
\NormalTok{)}

\FunctionTok{ggplot}\NormalTok{(goal\_to\_shot\_ratio, }\FunctionTok{aes}\NormalTok{(}\AttributeTok{x =}\NormalTok{ Competition, }\AttributeTok{y =}\NormalTok{ GoalToShotRatio, }\AttributeTok{fill =}\NormalTok{ Competition)) }\SpecialCharTok{+}
  \FunctionTok{geom\_bar}\NormalTok{(}\AttributeTok{stat =} \StringTok{"identity"}\NormalTok{) }\SpecialCharTok{+}
  \FunctionTok{geom\_text}\NormalTok{(}\FunctionTok{aes}\NormalTok{(}\AttributeTok{label =}\NormalTok{ GoalToShotRatio), }\AttributeTok{vjust =} \SpecialCharTok{{-}}\FloatTok{0.5}\NormalTok{) }\SpecialCharTok{+}
  \FunctionTok{labs}\NormalTok{(}\AttributeTok{title =} \StringTok{"Ratio Buts/Tirs selon la compétition"}\NormalTok{,}
       \AttributeTok{x =} \StringTok{"Compétition"}\NormalTok{, }\AttributeTok{y =} \StringTok{"Ratio Buts/Tirs"}\NormalTok{) }\SpecialCharTok{+}
  \FunctionTok{scale\_fill\_manual}\NormalTok{(}\AttributeTok{values =} \FunctionTok{c}\NormalTok{(}\StringTok{"FA Cup"} \OtherTok{=} \StringTok{"blue"}\NormalTok{, }\StringTok{"Premier League"} \OtherTok{=} \StringTok{"red"}\NormalTok{)) }\SpecialCharTok{+}
  \FunctionTok{theme\_minimal}\NormalTok{()}
\end{Highlighting}
\end{Shaded}

\includegraphics{rapport_files/figure-latex/unnamed-chunk-37-1.pdf}

\subsubsection{Analyse des Résultats}\label{analyse-des-ruxe9sultats-5}

Le ratio buts/tirs est légèrement plus élevé en FA Cup (0,105) par
rapport à la Premier League (0,1), malgré le nombre moyen de buts par
match étant plus élevé en FA Cup (1,77) comparé à la Premier League
(1,37). Cette observation suggère que les équipes en coupe ont été
légèrement plus efficaces pour convertir leurs tirs en buts, malgré un
nombre global de buts par match plus élevé. Plusieurs facteurs peuvent
expliquer cette tendance, notamment une concentration accrue sur
l'attaque dans les matchs à élimination directe, une pression réduite
par rapport à la Premier League et la qualité variable des adversaires
rencontrés.

\paragraph{Hypothèses et
Explications}\label{hypothuxe8ses-et-explications-4}

L'hypothèse initiale suggère que les matchs de coupe adoptent une
approche plus prudente et tactique, ce qui pourrait se traduire par
moins de buts par match par rapport aux championnats où les équipes ont
une saison entière pour s'adapter et se projeter.

Cependant, les résultats de l'analyse contredisent partiellement cette
hypothèse. Bien que la différence de nombre moyen de buts par match
entre la FA Cup et la Premier League ne soit pas aussi significative que
prévu, d'autres aspects des performances des équipes varient en fonction
de la compétition.

Par exemple, les données révèlent que le pourcentage de clean sheets est
plus élevé en Premier League, ce qui indique une meilleure efficacité
défensive dans cette compétition où les équipes ont plus de temps pour
se préparer et s'adapter tactiquement. De plus, le pourcentage d'arrêts
des gardiens est également supérieur en Premier League, suggérant une
pression défensive plus importante dans cette compétition.

Ces résultats mettent en lumière les nuances des performances des
équipes dans différentes compétitions et soulignent l'importance de
considérer plusieurs aspects du jeu pour comprendre pleinement les
différences entre les matchs de coupe et les championnats. Bien que
l'hypothèse initiale puisse ne pas être entièrement confirmée, l'analyse
fournit des insights précieux sur les stratégies et les performances des
équipes dans des contextes de compétition différents.

\section{Conclusion}\label{conclusion-2}

Les analyses approfondies des différences entre les championnats et les
coupes ont révélé des variations significatives dans les performances
des équipes, à la fois sur le plan offensif et défensif. Bien que
certaines conclusions aient contredit notre hypothèse initiale selon
laquelle les matchs de coupe se caractériseraient par moins de buts,
elles ont également mis en lumière des aspects intéressants du jeu qui
méritent d'être explorés plus en détail.

Il est clair que la nature des matchs, l'enjeu et les adversaires jouent
un rôle crucial dans les stratégies adoptées par les équipes. Par
exemple, bien que le nombre moyen de buts par match puisse ne pas
différer considérablement entre la FA Cup et la Premier League, d'autres
indicateurs tels que le pourcentage de clean sheets et le pourcentage
d'arrêts des gardiens varient de manière significative entre les deux
compétitions.

Ces résultats soulèvent plusieurs questions intrigantes pour de futures
recherches. Par exemple, comment les équipes ajustent-elles leurs
stratégies tactiques en fonction du format de la compétition et de la
qualité de l'adversaire ? Dans quelle mesure la fatigue et la pression
influencent-elles les performances des équipes dans des matchs à
élimination directe par rapport à une saison entière ?

En outre, cette étude met en évidence la nécessité de considérer
plusieurs aspects du jeu pour comprendre pleinement les différences
entre les matchs de coupe et les championnats. Il serait intéressant
d'explorer davantage l'impact des facteurs extérieurs tels que la météo,
les blessures et les enjeux psychologiques sur les performances des
équipes dans différentes compétitions.

En conclusion, cette étude fournit une base solide pour une analyse plus
approfondie des stratégies et des performances des équipes dans le
football professionnel. Les découvertes réalisées ouvrent la voie à de
nouvelles investigations pour mieux comprendre les dynamiques complexes
qui régissent le monde du football.

\section{Appliquer la fonction à chaque URL d'équipe de la FA
Cup}\label{appliquer-la-fonction-uxe0-chaque-url-duxe9quipe-de-la-fa-cup}

fa\_cup\_team\_results\_list \textless- lapply(fa\_cup\_team\_urls,
get\_fa\_cup\_team\_results)

\section{Combiner les résultats de la FA
Cup}\label{combiner-les-ruxe9sultats-de-la-fa-cup}

fa\_cup\_team\_results \textless- do.call(rbind,
fa\_cup\_team\_results\_list)

\section{Collecter les URLs des équipes en Premier League pour la saison
2022-2023}\label{collecter-les-urls-des-uxe9quipes-en-premier-league-pour-la-saison-2022-2023}

pl\_team\_urls \textless-
fb\_teams\_urls(``\url{https://fbref.com/en/comps/9/2022-2023/2022-2023-Premier-League-Stats}'')

\section{Fonction pour extraire les résultats de
championnat}\label{fonction-pour-extraire-les-ruxe9sultats-de-championnat}

get\_pl\_team\_results \textless- function(team\_url) \{ results
\textless- fb\_team\_match\_results(team\_url) results \textless-
results \%\textgreater\% mutate(Team = gsub(``.*/``,''``, team\_url),
Team = gsub(''-Stats'', ````, Team), Competition =''Premier League'', GF
= as.numeric(GF), GA = as.numeric(GA)) \%\textgreater\% select(Team,
Date, Opponent, Result, Venue, GF, GA, Comp = Competition)
return(results) \}

\section{Appliquer la fonction aux URLs de Premier
League}\label{appliquer-la-fonction-aux-urls-de-premier-league}

pl\_team\_results\_list \textless- lapply(pl\_team\_urls,
get\_pl\_team\_results)

\section{Combiner les résultats de la Premier
League}\label{combiner-les-ruxe9sultats-de-la-premier-league}

pl\_team\_results \textless- do.call(rbind, pl\_team\_results\_list)

\section{Fusionner les données de la FA Cup et de la Premier
League}\label{fusionner-les-donnuxe9es-de-la-fa-cup-et-de-la-premier-league}

all\_results \textless- rbind(fa\_cup\_team\_results, pl\_team\_results)

\section{Calculer les moyennes de buts marqués pour chaque
compétition}\label{calculer-les-moyennes-de-buts-marquuxe9s-pour-chaque-compuxe9tition}

average\_goals \textless- all\_results \%\textgreater\% group\_by(Comp)
\%\textgreater\% summarise(AverageGoals = mean(GF, na.rm = TRUE))

\section{Visualisation des moyennes de buts
marqués}\label{visualisation-des-moyennes-de-buts-marquuxe9s-1}

ggplot(average\_goals, aes(x = Comp, y = AverageGoals, fill = Comp)) +
geom\_col() + labs(title = ``Moyenne des buts marqués par match en FA
Cup vs Premier League'', x = ``Compétition'', y = ``Moyenne des buts
marqués par match'') + theme\_minimal() + theme(legend.position =
``none'')

\section{Afficher les résultats}\label{afficher-les-ruxe9sultats}

print(average\_goals)

\subsubsection{Interprétation et
conclusion}\label{interpruxe9tation-et-conclusion}

L'analyse montre que le nombre moyen de buts par match diffère entre les
compétitions de la FA Cup et de la Premier League. Cette information
pourrait suggérer différentes approches tactiques entre les matchs de
coupe et de championnat. Par exemple, les équipes pourraient adopter des
stratégies plus offensives ou défensives selon l'importance ou le format
du match.

\subsubsection{Questions pour approfondir
l'analyse}\label{questions-pour-approfondir-lanalyse}

Cette observation initiale mène à plusieurs questions intéressantes pour
des recherches futures :

\begin{itemize}
\tightlist
\item
  Comment les styles de jeu diffèrent-ils entre les matchs de coupe et
  de championnat au niveau des équipes individuelles ?
\item
  Y a-t-il une corrélation entre la fréquence des changements de joueurs
  et les performances dans les matchs de coupe par rapport au
  championnat ?
\item
  Quel est l'impact de la pression des matchs à élimination directe sur
  les performances des équipes en termes de buts marqués et encaissés ?
\end{itemize}

\end{document}
